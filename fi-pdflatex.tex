%%%%%%%%%%%%%%%%%%%%%%%%%%%%%%%%%%%%%%%%%%%%%%%%%%%%%%%%%%%%%%%%%%%%
%% I, the copyright holder of this work, release this work into the
%% public domain. This applies worldwide. In some countries this may
%% not be legally possible; if so: I grant anyone the right to use
%% this work for any purpose, without any conditions, unless such
%% conditions are required by law.
%%%%%%%%%%%%%%%%%%%%%%%%%%%%%%%%%%%%%%%%%%%%%%%%%%%%%%%%%%%%%%%%%%%%

\documentclass[
  digital, %% This option enables the default options for the
           %% digital version of a document. Replace with `printed`
           %% to enable the default options for the printed version
           %% of a document.
  twoside, %% This option enables double-sided typesetting. Use at
           %% least 120 g/m² paper to prevent show-through. Replace
           %% with `oneside` to use one-sided typesetting; use only
           %% if you don’t have access to a double-sided printer,
           %% or if one-sided typesetting is a formal requirement
           %% at your faculty.
  table,   %% This option causes the coloring of tables. Replace
           %% with `notable` to restore plain LaTeX tables.
  nolof,     %% This option prints the List of Figures. Replace with
           %% `nolof` to hide the List of Figures.
  nolot,     %% This option prints the List of Tables. Replace with
           %% `nolot` to hide the List of Tables.
  %% More options are listed in the user guide at
  %% <http://mirrors.ctan.org/macros/latex/contrib/fithesis/guide/mu/fi.pdf>.
]{fithesis3}
%% The following section sets up the locales used in the thesis.
\usepackage[resetfonts]{cmap} %% We need to load the T2A font encoding
%\usepackage[T1,T2A]{fontenc}  %% to use the Cyrillic fonts with Russian texts.
\usepackage[
  main=english, %% By using `czech` or `slovak` as the main locale
                %% instead of `english`, you can typeset the thesis
                %% in either Czech or Slovak, respectively.
  english, slovak, czech%german, russian, czech, slovak %% The additional keys allow
]{babel}        %% foreign texts to be typeset as follows:
%%
%%   \begin{otherlanguage}{german}  ... \end{otherlanguage}
%%   \begin{otherlanguage}{russian} ... \end{otherlanguage}
%%   \begin{otherlanguage}{czech}   ... \end{otherlanguage}
%%   \begin{otherlanguage}{slovak}  ... \end{otherlanguage}
%%
%% For non-Latin scripts, it may be necessary to load additional
%% fonts:
\usepackage{paratype}
\def\textrussian#1{{\usefont{T2A}{PTSerif-TLF}{m}{rm}#1}}
%%
%% The following section sets up the metadata of the thesis.
\thesissetup{
    date          = \the\year/\the\month/\the\day,
    university    = mu,
    faculty       = fi,
    type          = mgr,
    author        = Juraj S\'{i}\v{c},
    gender        = m,
    advisor       = Jan Strej\v{c}ek,
    title         = {Solving DQBF using BDDs},
    TeXtitle      = {Solving DQBF using BDDs},
    keywords      = {dependency quantified Boolean formula, DQBF, binary decision diagram, BDD, satisfiability, solver},
    TeXkeywords   = {dependency quantified Boolean formula, DQBF, binary decision diagram, BDD, satisfiability, solver},
    abstract      = {%
      This is the abstract of my thesis, which can

      span multiple paragraphs.
    },
    thanks        = {%
      These are the acknowledgements for my thesis, which can

      span multiple paragraphs.
    },
    bib           = bibliography.bib
    %% Uncomment the following line (by removing the % symbol at
    %% the beginning) and replace `assignment.pdf` with the
    %% filename of your scanned thesis assignment.
    % assignment    = assignment.pdf,
}
\usepackage{makeidx}      %% The `makeidx` package contains
\makeindex                %% helper commands for index typesetting.
%% These additional packages are used within the document:
\usepackage{paralist} %% Compact list environments
\usepackage{amsmath}  %% Mathematics
\usepackage{amsthm}
\usepackage{amsfonts}
\usepackage{url}      %% Hyperlinks
\usepackage{markdown} %% Lightweight markup
\usepackage{tabularx} %% Tables
\usepackage{tabu}
\usepackage{booktabs}
\usepackage{listings} %% Source code highlighting
\lstset{
  basicstyle      = \ttfamily,
  identifierstyle = \color{black},
  keywordstyle    = \color{blue},
  keywordstyle    = {[2]\color{cyan}},
  keywordstyle    = {[3]\color{olive}},
  stringstyle     = \color{teal},
  commentstyle    = \itshape\color{magenta},
  breaklines      = true,
}
\usepackage{floatrow} %% Putting captions above tables
\floatsetup[table]{capposition=top}
%% The following code fixes the rendering of BibLaTeX ISO 690
%% references in old TeX Live (such as the one at Overleaf).
\thesisload
\makeatletter
\def\thesis@biblatexiso@fix@package{iso-numeric.bbx}
\def\thesis@biblatexiso@fix@end{\relax}
\newif\ifthesis@biblatexiso@fix@
\thesis@biblatexiso@fix@false
\def\thesis@biblatexiso@fix@next#1,{%
  \def\thesis@biblatexiso@fix@current{#1}%
  \ifx\thesis@biblatexiso@fix@current\thesis@biblatexiso@fix@package
    \thesis@biblatexiso@fix@true
  \fi
  \ifx\thesis@biblatexiso@fix@current\thesis@biblatexiso@fix@end
    \expandafter
    \@gobble
  \fi
  \thesis@biblatexiso@fix@next
}
\expandafter\expandafter\expandafter\thesis@biblatexiso@fix@next\@filelist,\relax,
\ifthesis@biblatexiso@fix@
  \defbibenvironment{bibliography}
    {\list%
       {\MethodFormat}%
       {\setlength{\labelwidth}{\labelnumberwidth}%
        \setlength{\leftmargin}{\labelwidth}%
        \setlength{\labelsep}{\biblabelsep}%
        \addtolength{\leftmargin}{\labelsep}%
        \setlength{\itemsep}{\bibitemsep}%
        \setlength{\parsep}{\bibparsep}}%
        \renewcommand*{\makelabel}[1]{\hss##1}
        }%
    {\endlist}%
  {\item}%
\fi
\makeatother

%%%%%%%%%%%%%%%%%%%%%%%%%%%%%%%%%%%%%%%%%%%%%%%%%%%%%%%%%
%%%%%%%%%%%%%%%%%%%%%%%%%%%%%%%%%%%%%%%%%%%%%%%%%%%%%%%%%
%%%%%% The packages I use, my definitions, etc. %%%%%%%%%
%%%%%%%%%%%%%%%%%%%%%%%%%%%%%%%%%%%%%%%%%%%%%%%%%%%%%%%%%
%%%%%%%%%%%%%%%%%%%%%%%%%%%%%%%%%%%%%%%%%%%%%%%%%%%%%%%%%

% for writing nice pseudocode...
\usepackage{algpseudocode}
% ...and a float wrapper for them
\usepackage{algorithm}

\usepackage{mathtools}

% for checking for | in definition of set
\usepackage{xstring}

% for \set{}
\usepackage{braket}
% for notation { x | P(x)} we will use command \setbuilder
\let\setbuilder\set
% for {1,2,3}
\newcommand{\simpleset}[1]{\{{#1}\}}
% based on whether | is in the argument either use set builder notation or simple set notation
\renewcommand{\set}[1]{\normalexpandarg\IfSubStr{#1}{|}{\setbuilder{#1}}{\simpleset{#1}}}
%\renewcommand{\set}[1]{\simpleset{#1}}

% for semantic symbols [[ and ]]
\usepackage{stmaryrd}

% for nicer looking substitutions in formulas
\usepackage{nicefrac}

% for writing out rules of DQBF in NNF
\usepackage{ebproof}

% for writing left aligned text in figures
%\usepackage{caption}

%% We will define several mathematical sectioning commands.
\newtheorem{theorem}{Theorem}[chapter] %% The numbering of theorems
                               %% will be reset after each chapter.
\newtheorem{lemma}[theorem]{Lemma}         %% The numbering of lemmas
\newtheorem{corollary}[theorem]{Corollary} %% and corollaries will
                               %% share the counter with theorems.
\theoremstyle{definition}
\newtheorem{definition}{Definition}
\newtheorem{example}{Example}
\theoremstyle{remark}
\newtheorem*{remark}{Remark}

% [[ #1 ]]
\newcommand{\seman}[1]{\left\llbracket {#1} \right\rrbracket}
% substitution for formulas: [phi'/phi] - \substitute{phi}{phi'}
\newcommand{\substitute}[2]{\left[ \nicefrac{#2}{#1} \right]}
% the set of all boolean formulas over #1
\newcommand{\BF}[1]{BF(#1)}
% the set of all dependency quantified boolean formulas over #1
\newcommand{\DQBF}[1]{\mathbf{\Phi}_{#1}^{\text{\tiny dnnf}}}
% the set of all valuations over #1
\newcommand{\valtns}[1]{\mathbf{A}_{#1}}
% the set of all Boolean functions over #1
\newcommand{\BFuncs}[1]{\mathbb{F}_{#1}}

% the set of existential variables for formula #1
\newcommand{\evars}[1]{V_{#1}^{\exists}}
% the set of universal variables for formula #1
\newcommand{\uvars}[1]{V_{#1}^{\forall}}
% the set of free variables for formula #1
\newcommand{\fvars}[1]{V_{#1}^{\text{free}}}

% support set of boolean function #1
\newcommand{\supp}[1]{\text{supp}(#1)}

% set of skolem function candidates for formula #1
\newcommand{\Scands}[1]{\mathbb{S}_{#1}}

% logical equality (like \lor or \land)
\newcommand{\lequal}{\Leftrightarrow}

%\renewcommand{\phi}{\varphi}

% for fixing overflowing in bibliography
\hyphenation{Chris-toph}
%\hyphenation{Mat-thias}

%%%%%%%%%%%%%%%%%%%%%%%%%%%%%%%%%%%%%%%%%%%%%%%%%%%%%%%%%
%%%%%%%%%%%%%%%%%%%%%%%%%%%%%%%%%%%%%%%%%%%%%%%%%%%%%%%%%

\begin{document}
\chapter*{Introduction}
\addcontentsline{toc}{chapter}{Introduction}

\chapter{Theory}
In this chapter we give the definitions and some theoretical results of dependency quantified Boolean formulas (DQBFs) and binary decision diagrams (BDDs). For DQBFs we build up from Boolean formulas (BFs) to which we add quantifiers resulting in quantified Boolean formulas (QBFs). After this, we add explicit dependencies to the definition of QBFs which results in DQBFs. For these we give two forms, one that does not allow quantifiers inside the formula and one that does. For the latter we give some theoretical results which allow us to transform DQBFs which we then use for developing our solver in Chapter~\ref{chap:algorithm}. In the last section of this chapter we define BDDs and give some of their properties.% needed theory for our problem. We definition of dependency quantified Boolean formulas??? and BDDs?? citacie odkial vychadzam??

%We define dependency quantified Boolean formulas building up from Boolean formulas to which we add quantifications followed by adding dependencies for these quantifications. We also give a 

\section{Boolean Formulas}
%definicia boolean formula (rekurzivna)

We first start with defining Boolean formulas. For these we need a set of variables $V = \set{x_1, \dots, x_n}$ for which we can define a valuation function $v\colon V \to \{0,1\}$ that says whether each variable is true or false. We then define Boolean formulas as variables connected by logical operatives \emph{and} (${\land}$), \emph{or} (${\lor}$) and \emph{negation} (${\neg}$).

\begin{definition}
Let $V = \set{x_1, \dots, x_n}$ be a set of variables. The set of \emph{Boolean formulas} (BFs) over $V$, denoted by $\BF{V}$, is defined as the smallest set fulfilling these conditions:
\begin{itemize}
    \item $0,1 \in \BF{V}$,
    \item $x \in \BF{V}$ if $x \in V$,
    \item $(\phi_1 \land \phi_2) \in \BF{V}$ if $\phi_1, \phi_2 \in \BF{V}$,
    \item $(\phi_1 \lor \phi_2) \in \BF{V}$ if $\phi_1, \phi_2 \in \BF{V}$,
    \item $\neg\phi \in \BF{V}$ if $\phi \in \BF{V}$.
\end{itemize}
\end{definition}
By $\phi\substitute{\phi_1}{\phi_2}$, where $\phi,\phi_1,\phi_2 \in \BF{V}$, we denote the BF where each occurrence of $\phi_1$ in $\phi$ is replaced by $\phi_2$.

A function $v\colon V \to \set{0,1}$ is called a \emph{valuation} over the set of variables $V$. A set of all valuations over $V$ is denoted by $\valtns{V}$. An \emph{evaluation} of $\phi \in \BF{V}$ in a valuation $v \in \valtns{V}$, denoted $v(\phi)$, is defined as %can extend this valuation to $\BF{V}$%, $v\colon \BF{V} \to \set{0,1}$ where
\begin{itemize}
    \item if $\phi = 0$ then $v(\phi) = 0$,
    \item if $\phi = 1$ then $v(\phi) = 1$,
    \item if $\phi = x_i$ where $x_i \in V$, then $v(\phi) = v(x_i)$,
    \item if $\phi = \neg\phi'$, then $v(\phi) = 1$ if $v(\phi') = 0$ otherwise $v(\phi) = 0$,
    \item if $\phi = (\phi_1 \land \phi_2)$, then $v(\phi) = 1$ if both $v(\phi_1) = 1$ and $v(\phi_2) = 1$, otherwise $v(\phi) = 0$ and
    \item if $\phi = (\phi_1 \lor \phi_2$), then $v(\phi) = 1$ if either $v(\phi_1) = 1$ or $v(\phi_2) = 1$, otherwise $v(\phi) = 0$.
\end{itemize}
Each $\phi \in \BF{V}$ represents a \emph{Boolean function} $f_{\phi}\colon \valtns{V} \to \set{0,1}$ over $V$ such that $f_{\phi}(v) = v(\phi)$. The set of all Boolean functions over $V$ is denoted by $\BFuncs{V}$. The \emph{support set} $\supp{f}$ of Boolean function $f$ is defined as the set of variables occurring in a BF $\phi$ where from all the BFs representing $f$, $\phi$ has the smallest number of variables occurring in it. That is, changing the values of variables that are not in the support set do not change the output value of $f$.
%We will abuse notation and use $v(\phi)$ for extended $v$????? 
%mozno napisat ako ze ten extend nejak oznacime (napr u_v) ale ak je to jasne tak abusneme notation a budeme pod $v(\phi)$ mysliet $u_v(\phi)$


An important notion for BFs is \emph{satisfiability}. We say that $\phi \in \BF{V}$ is \emph{satisfiable} if there exists valuation $v \in \valtns{V}$ in which $\phi$ is true, that is $v(\phi) = 1$. Let $\phi_1, \phi_2 \in \BF{V}$. We say that they are \emph{equivalent}, denoted $\phi_1 \equiv \phi_2$, if for all valuations $v$ it holds that $v(\phi_1) = v(\phi_2)$. If they are either both satisfiable or they are both unsatisfiable, we call them \emph{equisatisfiable}, denoted $\phi_1 \approx \phi_2$. %Notice that equisatisfiability of two formulas is just talking about their satisfiability, which means that these formulas can be over different sets of variables.

\begin{example}
\label{ex:BF:prenex}
Let
\[\phi = (x_1 \land x_2) \lequal (y_1 \lequal y_2)\]
be BF where by $y_1 \lequal y_2$ we mean the formula $(y_1 \land y_2) \lor (\neg y_1 \land \neg y_2)$ (similarly for the ${\lequal}$ between $x_1\land x_2$ and $y_1 \lequal y_2$). This formula says that if both $x_1$ and $x_2$ are true, then $y_1$ and $y_2$ must be same, otherwise they must be different. It is satisfiable, because in valuation $v$ such that $v(x_1) = v(x_2) = v(y_1) = v(y_2) = 1$ it holds that $\phi$ is evaluated to $v(\phi)=1$.
%This formula is also equisatisfiable to formula 
%\[\phi' = (x \land y) \lor \neg x\] 
%because $\phi'$ is also satisfiable: in valuation $v'$ where $v'(x) = v'(y) = 1$ it holds that $v'(\phi') = 1$. Notice that even though $\phi$ and $\phi'$ have different variables, the equisatisfiability still holds. However, $\phi \not= \phi'$ because if we add to $v'$ that $v'(x_1) = v'(x_2) = v'(y_1) = 1$ and $v'(y_2) = 0$ we get that $v'(\phi) \not= v'(\phi')$.
\end{example}

%definovat negation normal form + cnf
\subsection{Conjuctive Normal Form}
\label{sec:BF:CNF}
To work with BFs it is usually easier to have them in some special form. The first one we define is called conjuctive normal form.

\begin{definition}
We say that boolean formula $\phi \in \BF{V}$ is in \emph{conjuctive normal form} (CNF) if there exist $\phi_1,\dots,\phi_n \in \BF{V}$ where
\[\phi = \phi_1 \land \dots \land \phi_n\]
and
\[\phi_i = l_{i1} \lor \dots \lor l_{i{m_i}}\]
for each $i = 1, \dots, n$ and for each $l_{ij}$ where $j \in \set{1,\dots,m_i}$ either $l_{ij} = x$ or $l_{ij} = \neg x$ for some $x \in V$.
\end{definition}
Formulas $\phi_1,\dots,\phi_n$ are called clauses and $l_{ij}$ are called literals. That is BF is in CNF if it is a conjuction of clauses which are disjuncts of literals, where literal is either a variable or negation of a variable.

\begin{example}
The formula
\[\phi_{CNF} = (x \lor \neg y) \land (\neg x \lor y)\]
is in CNF with clauses $(x \lor \neg y)$ and $(\neg x \lor y)$ and literals $x,\neg y, \neg x$ and $y$.
\end{example}

For every BF $\phi$ there exist an equivalent formula in CNF which can be exponentially larger than $\phi$. However, by using Tseytin transformation~\cite{Tseitin} we can create an equisatisfiable BF $\phi_{CNF}$ which is only polynomially larger than the original BF $\phi$.

\subsection{Negation Normal Form}
\label{sec:BF:NNF}
We say that BF $\phi$ is in negation normal form (NNF) if negation occurs only in front of the variable. Each BF $\phi$ can be easily transformed into an equivalent BF $\phi'$ in NNF by applying these three rules:
\begin{enumerate}
    \item $\neg(\phi_1 \land \phi_2) \equiv (\neg\phi_1 \lor \neg\phi_2)$ 
    \item $\neg(\phi_1 \lor \phi_2) \equiv (\neg\phi_1 \land \neg\phi_2)$
    \item $\neg\neg x \equiv x$
\end{enumerate}
\begin{example}
Formula $\neg (x_1 \land (x_2 \lor \neg x_3))$ that is not in NNF is by using the first rule equal to $(\neg x_1 \lor \neg(x_2 \lor \neg x_3))$ which is by using the second rule equal to $(\neg x_1 \lor (\neg x_2 \land \neg\neg x_3))$. Finally, by using the third rule we get $(\neg x_1 \lor (\neg x_2 \land x_3))$ which is in NNF.
\end{example}
%toto asi zadefinujem pre QBF/DQBF ak vobec - alebo mozno to treba fakt tu, ked budem chciet hovorit ze HQSpre meni formulu????

\section{Quantified Boolean Formulas}
Having defined BFs, we can move to the next step on the way to defining DQBF by adding quantifiers. We add existential (${\exists}$) and universal (${\forall}$) quantifiers which are bounded to variables resulting in \emph{quantified Boolean formulas} (QBFs). Usually, definitions of QBFs allow quantifiers everywhere inside formula but we only give definition of one special form of QBFs called prenex normal form. In this form, QBFs allow quantifiers only at the beginning of the formula. However, this does not change the expressibility as it is possible to transform every QBF into prenex normal form~\cite{prenexingQBFs}.

%definicia QBF (ak budem robit aj non-prenex DQBF tak rekurzivna + nahradzovanie, inak prenex)
\begin{definition}
Let $V = \set{x_1,\dots,x_n}$ be a set of variables and $\phi \in \BF{V}$. A \emph{quantified Boolean formula} (QBF) $\psi$ over $V$ in closed prenex form is given by
\[\psi \coloneqq Q_1 x_1 Q_2 x_2 \dots Q_n x_n \phi\]
where $Q_i \in \set{{\exists}, {\forall}}$ and $x_i \in V$ are distinct for all $i \in \set{1,2,\dots,n}$.
\label{def:QBF}
\end{definition}

The BF $\phi$ is called the \emph{matrix} of $\psi$ and $Q_1 x_1 Q_2 x_2 \dots Q_n$ is called the \emph{quantifier prefix} of $\psi$ A variable $x_i$ is called \emph{existential} if $Q_i = \exists$ and \emph{universal} if $Q_i = \forall$. The set of all existential variables of QBF $\psi$ is denoted by $\evars{\psi}$ and the set of all universal variables of $\psi$ is denoted by $\uvars{\psi}$. We define a level function $l\colon V \to \mathbb{N}$ which maps to each variable the number of quantifiers occurring before the variable in the formula. That is the variable $x_i \in V$ from QBF $\psi$ from Definition~\ref{def:QBF} has level $l(x_i) = i$. We denote the set of variables with level less than some $n \in \mathbb{N}$ as $V_{\psi}^{<n} = \set{x \in V | l(x) < n}$.
%We call the variables $x_i$ such that $Q_i = \exists$ existential variables (denoted $V_{\psi}^{\exists}$) and the variables where $Q_i = \forall$ universal variables (denoted $V_{\psi}^{\forall}$). To each variable $x_i$ we assign a level (denoted $l(x_i)$), 

Again, we define satisfiability. Usually, satisfiability for QBFs is defined trough valuation function as in BFs case where evaluation of QBFs follows the same rules as BFs with two newly added rules for quantifiers:
\begin{itemize}
    \item $v(\forall x Q_i x_i \dots Q_n x_n \phi) = 1$ if both $v(Q_i x_i \dots Q_n x_n \phi\substitute{x}{0}) = 1$ and $v(Q_i x_i \dots Q_n x_n \phi\substitute{x}{1}) = 1$, otherwise it equals $0$,
    \item $v(\exists x Q_i x_i \dots Q_n x_n \phi) = 1$ if either $v(Q_i x_i \dots Q_n x_n \phi\substitute{x}{0}) = 1$ or $v(Q_i x_i \dots Q_n x_n \phi\substitute{x}{1}) = 1$, otherwise it equals $0$.
\end{itemize}
We follow different but equal definition based on skolemization, which replaces each existential variable $x$ by some Boolean function of universal variables with lower level than $x$. This is better suited to show a correspondence of QBFs and DQBFs when we add dependencies in the next section. A QBF $\psi$ from Definition~\ref{def:QBF} is satisfiable if for each $x \in V_{\psi}^{\exists}$ there exists a function $s_{x} \in \BFuncs{\uvars{\psi}\cap V_{\psi}^{<l(x)}}$ (called Skolem function) such that $\phi$, where every $x \in \evars{\psi}$ is replaced by some BF that represents $s_x(x)$, is evaluated to $1$ in every valuation $v\colon \uvars{\psi} \to \set{0,1}$.

\begin{example}
\label{ex:QBF:prenex}
Let 
\[\psi = \forall x_1 \forall x_2 \exists y_1  \exists y_2  : (x_1 \land x_2) \lequal (y_1 \lequal y_2)\]
be a QBF where the matrix is formula $\phi$ from Example~\ref{ex:BF:prenex}. This QBF is satisfiable because for all values of $x_1, x_2$ we can find values of $y_1$ and $y_2$ such that they are same if $x_1 = x_2 = 1$ and different otherwise. The Skolem functions for this formula are for example $s_{y_1}$ which is represented by BF $x_1 \land x_2$ and $s_{y_2}$ which is represented by BF $1$. If we replace existential variables with their Skolem function we get BF
\[(x_1 \land x_2) \lequal ((x_1 \land x_2) \lequal 1)\]
which is true in all valuations.
\end{example}

%definicia semantiky (+splnitelnost)
%Let $Q\BF{V}$ be the set of all quantified Boolean formulas over $V$ and $v\colon V \to \set{0,1}$ a valuation over the set of variables $V$. Again, we can extend $v$ to $Q\BF{V}$ where 
%\begin{itemize}
%    \item for $\phi \in \BF{V}$, $v(\phi)$ behaves like $v$ extended to $\BF{V}$ as explained in section TODO
%    \item $v(\forall x Q_2 x_2 \dots Q_n x_n \phi) = 1$ if both $v(Q_2 x_2 \dots Q_n x_n \phi) = 1$ and ... TODO definovat nahrazdovanie + doplnit do definicie BF 1 a 0 konstanty
%\end{itemize}
%mozno pisat skor ze pre \psi z QBF, v(psi) je definovana pre tvar \psi = itemize ...
%definovat bounded
%Notice that in this definition of semantics the variables that are bounded with some quantifier do not care what they are evaluated to. That is priklad It can be seen that if addded $\exists x$ to unbounded variables $x$ to the beginning of any QBF, we would not need any valuation etcetc

%We can also define satisfiability on QBF, similarly to BF. We say that QBF $\psi$ is satisfiable if there exists valuation $v$ such that $v(\psi) = 1$.

\section{Dependency Quantified Boolean Formulas}
Quantifiers have expanded the expressibility of BFs quite considerably. However, there is still one drawback of QBFs --- quantified variable $x$ depends on all variables that are quantified before $x$ in the formula. The question arises whether it is possible to give a definition of formula where the dependency relation is somehow explicitly given. This results in a definition of \emph{dependency quantified Boolean formulas} (DQBFs).

DQBFs allow existential variables to be non-linearly dependent on the set of universal variables by explicitly writing out the set of universal variables on which each existential variable is dependent. This set is then used as a support set of the skolem functions of the existential variables which allows for richer expresibility of DQBFs over QBFs.  

We start with the definition of DQBFs in prenex normal form on which we explain the basic notions and then we give definition of DQBFs in non-prenex negation normal form which allows quantifiers inside formulas. Following this, we give theorems that show which DQBFs are equisatisfiable and can be replaced with each other which can be used to push quantifiers inside the formula or to eliminate quantifiers.
%\begin{itemize}
%    \item pushing quantifiers inside formula,
%    \item elimination of quantified variables or
%    \item ????removal of universal variables from the dependency sets of existential variables.????
%\end{itemize}

\subsection{Prenex Normal Form}
The first definition of DQBF form called prenex normal form is an analogy of the prenex normal form of QBFs.


%definicia DQBF -ak robim nonprenex tak zacat s prenex ale vysvetlit ze pre jednoduchost zacneme s tym + semantika prenex + priklady
\begin{definition}
  Let $V = \set{x_1, \dots, x_n, y_1, \dots, y_m}$ be a set of variables and $\phi \in \BF{V}$ a BF over $V$. A \emph{dependency quantified Boolean formula} (DQBF) $\psi$ in \emph{prenex normal form} (PNF) is given by
  \[\psi \coloneqq \forall x_1 \dots \forall x_n \exists y_1 (D_1) \dots \exists y_m (D_m) \phi\]
  where $D_i \subseteq \set{x_1, \dots, x_n}$ for each $i \in \set{1,\dots,m}$ is a \emph{dependency set} of variable $y_i$.
  \label{def:DQBF:PNF}
\end{definition}
We call the BF $\phi$ the \emph{matrix} of $\psi$ and $\forall x_1 \dots \forall x_n \exists y_1 (D_1) \dots \exists y_m (D_m)$ the \emph{quantifier prefix} of $\psi$. We also call variables from $\evars{\psi} = \set{y_1,\dots,y_m}$ \emph{existential} and variables from $\uvars{\psi} = \set{x_1,\dots,x_n}$ \emph{universal}.



\begin{example}
Let us use an example from~\cite{DQBFProofSystem} to explain the meaning behind dependency sets. Let 
\[\psi = \forall x_1 \forall x_2 \exists y_1 (x_1) \exists y_2 (x_2) : (x_1 \land x_2) \lequal (y_1 \lequal y_2)\]
be a DQBF. The matrix is the BF $\phi$ from Example~\ref{ex:BF:prenex} and the quantifier prefix is very similar to the QBF from Example~\ref{ex:QBF:prenex} but now both $y_1$ and $y_2$ depend only on one variable, $x_1$ and $x_2$ respectively. The QBF was satisfiable but this DQBF is not. To show why, we can look at it as a game where $y_1$  and $y_2$ are trying to satisfy the formula and $x_1,x_2$ are trying to make it false. However, $y_1$ knows only how $x_1$ is behaving and similarly $y_2$ knows only what $x_2$ is doing. As explained in Example~\ref{ex:BF:prenex} the matrix says that if both $x_1$ and $x_2$ are true, then $y_1$ and $y_2$ should be same, otherwise they need to be different. However, this is not possible, because if for example both $y_1,y_2$ decide to be true when their respective universal variables are true then if one of the universal variable change, for example $x_1$ turns to false, then $y_2$ does not know about this and stays true. In that case $y_1$ has to change the value to false so they become different. The same thing happens to $y_2$ if $x_2$ changes its value to false. So we get behaviour where both existential variables just copy the behaviour of their universal variable. But at the end, if both $x_1$ and $x_2$ change their values to false, then the values of $y_1$ and $y_2$ are the same (false) which makes the formula $\psi$ false.
\end{example}

With the example in mind, we define satisfiability for DQBF in PNF similarly to how satisfiability is defined for QBFs with Skolem functions. DQBF in PNF $\psi$ from Definition~\ref{def:DQBF:PNF} is satisfiable if for each $y \in \evars{\psi}$ there exists a Skolem function $s_y \in \BFuncs{D_y}$ such that $\phi$, where each $y \in \evars{\psi}$ is replaced by a BF representing Boolean function $s_y(y)$, evaluates to $1$ in all valuations $v\colon \uvars{\psi} \to \set{0,1}$.

\begin{example}
The possible Skolem functions $s_{y_1}$ for existential variable $y_1$ from previous Example are those that are represented by BFs $0$, $1$, $x_1$ or $\neg x_1$. For $y_2$ the Skolem functions $s_{y_2}$ can be those that are represented by BFs $0$, $1$, $x_2$ or $\neg x_2$. However, we cannot choose a pair $s_{y_1}$ and $s_{y_2}$ such that replacing $y_1$ and $y_2$ we get a BF which is true in all valuations. In the previous Example we actually used $s_{y_1}$ represented by $x_1$ and $s_{y_2}$ represented by $x_2$ where we get BF
\[(x_1 \land x_2) \lequal (x_1 \lequal x_2)\]
which is obviously not true in all valuations.
\end{example}

\begin{remark}
QBFs in PNF can be seen as a special case of DQBFs in PNF where the dependency sets are linearly ordered. That is a DQBF $\psi$ is a QBF if for each two dependency sets $D_{y_1}$ and $D_{y_2}$ in $\psi$ it holds that either $D_{y_1} = D_{y_2}$ or $D_{y_1} \subseteq D_{y_2}$ or $D_{y_2} \subseteq D_{y_1}$.
\end{remark}

%TODO: asi povedat ze PCNF je DQBF in PNF kde BF je in CNF

\subsection{Negation Normal Form}
%tu dat nonprenex ak robim, semantika + priklady + dokazy o nahradzovani a eliminovani kvantifikatorov - dat dokaz ze ked equality tak to ide nahradit (kvoli tomu ze maju rovnake hodnoty volnych premennych)

In this section we give a definition of normal form of DQBF that is non-prenex. This allows quantifiers inside the formula. However, negation is only allowed in front of variables. This restriction stems from the fact that prenex DQBFs are not syntactically closed under negation~\cite{NegationOfDQBFs}. %, therefore negation in front of some quantified subformula would make no sense.
This form is also not closed which means it allows free variables. These variables are not bounded by any quantifier thus the definition of satisfiability has to take them into consideration. The definitions in this section are from~\cite{HQSquantifierLocalization}.


\begin{figure}[htp]
    \renewcommand{\arraystretch}{2}
    %\begin{centering}
    \hspace*{-0.7cm}\begin{tabular}{ccccc}
        \multicolumn{2}{c}{rule} & $V_{.}^{\exists}$ & $V_{.}^{\forall}$ & $V_{.}^{\text{free}}$\\\hline
        {\begin{prooftree}
            \infer0{0 \in \DQBF{V}}
        \end{prooftree}}
        & {\begin{prooftree}
            \infer0{1 \in \DQBF{V}}
        \end{prooftree}}
        & $\emptyset$ & $\emptyset$ & $\emptyset$\\[5mm]
        {\begin{prooftree}
            \hypo{v \in V}
            \infer1{v \in \DQBF{V}}
        \end{prooftree}}
        %& $\emptyset$ & $\emptyset$ & $\set{v}$\\[5mm]
        & {\begin{prooftree}
            \hypo{v \in V}
            \infer1{\neg v \in \DQBF{V}}
        \end{prooftree}}
        & $\emptyset$ & $\emptyset$ & $\set{v}$\\[5mm]
        \multicolumn{2}{c}{{\begin{prooftree}
            \hypo{\psi_1 \in \DQBF{V}}
            \hypo{\psi_2 \in \DQBF{V}}
            \hypo{(\ref{disconjrule})}
            \infer3{(\psi_1 \land \psi_2) \in \DQBF{V}}
        \end{prooftree}}}
        & $\evars{\psi_1} \cup \evars{\psi_2}$ & $\uvars{\psi_1} \cup \uvars{\psi_2}$ & $\fvars{\psi_1} \cup \fvars{\psi_2}$\\[5mm] 
        \multicolumn{2}{c}{{\begin{prooftree}
            \hypo{\psi_1 \in \DQBF{V}}
            \hypo{\psi_2 \in \DQBF{V}}
            \hypo{(\ref{disconjrule})}
            \infer3{(\psi_1 \lor \psi_2) \in \DQBF{V}}
        \end{prooftree}}}
        & $\evars{\psi_1} \cup \evars{\psi_2}$ & $\uvars{\psi_2} \cup \uvars{\psi_2}$ & $\fvars{\psi_1} \cup \fvars{\psi_2}$\\[5mm]
        \multicolumn{2}{c}{{\begin{prooftree}
            \hypo{\psi \in \DQBF{V}}
            \hypo{v \in V \setminus (\evars{\psi} \cup \uvars{\psi})}
            \hypo{(\ref{existrule})}
            \infer3{\exists v(D_v):\psi^{-v} \in \DQBF{V}}
        \end{prooftree}}}
        & $\evars{\psi} \cup \set{v}$ & $\uvars{\psi}$ & $\fvars{\psi} \setminus \set{v}$ \\[5mm]
        \multicolumn{2}{c}{{\begin{prooftree}
            \hypo{\psi \in \DQBF{V}}
            \hypo{v \in V \setminus (\evars{\psi} \cup \uvars{\psi})}
            \infer2{\forall v:\psi \in \DQBF{V}}
        \end{prooftree}}}
        & $\evars{\psi}$ & $\uvars{\psi} \cup \set{v}$ & $\fvars{\psi} \setminus \set{v}$\\[5mm]
    \end{tabular}
    %\end{centering}
    \flushleft{where (\ref{disconjrule}) refers to}
    \begin{multline}
        \tag{1} \label{disconjrule}
        (\evars{\psi_1} \cup \uvars{\psi_1}) \cap (\evars{\psi_2} \cup \uvars{\psi_2}) = \emptyset \land \fvars{\psi_1} \cap (\evars{\psi_2} \cup \uvars{\psi_2}) = \emptyset\\
        \land \fvars{\psi_2} \cap (\evars{\psi_1} \cup \uvars{\psi_1}) = \emptyset
    \end{multline}
    \flushleft{and (\ref{existrule}) refers to}
    \begin{equation}
        \tag{2} \label{existrule}
        D_v \subseteq V \setminus (\evars{\psi} \cup \uvars{\psi} \cup \set{v})
    \end{equation}
    \caption{Rules defining the syntax of DQBFs in NNF (based on~\cite{HQSquantifierLocalization}).}
    \label{fig:DQBFrules}
\end{figure}

\begin{definition}
  Let $V$ be the set of variables. Let $\psi^{-v}$ result from $\psi$ where $v$ is removed from every dependency set in $\psi$. The set $\DQBF{V}$ of \emph{DQBFs in negation normal form} (NNF) over $V$ as well as their existential, universal and free variables are defined by Figure~\ref{fig:DQBFrules}. The set $\DQBF{V}$ is defined to be the smallest set satisfying these rules. We set $V_{\psi} = \evars{\psi}\cup\uvars{\psi}\cup\fvars{\psi}$.
\end{definition}
\begin{remark}
In~\cite{HQSquantifierLocalization} the last two rules in Figure~\ref{fig:DQBFrules} have condition $v \in \fvars{\psi}$ instead of $v \in V \setminus (\evars{\psi} \cup \uvars{\psi})$. This condition is too strong and does not allow some formulas which are valid. For example, as we show in the next section, we can transform DQBFs by pushing quantifiers inside the formula and this way we can end up with a formula $\forall x : \psi$ where $x \not\in V_{\psi}$. However, according to the definition in~\cite{HQSquantifierLocalization} this would not be a valid formula.
\end{remark}

\begin{remark}
Notice that each DQBF $\psi$ in PNF can be transformed into DQBF $\psi'$ in NNF by transforming the matrix of $\psi$ to NNF as explained in Section~\ref{sec:BF:NNF}.
\end{remark}

By $\psi\substitute{\psi_1}{\psi_2}$ where $\psi, \psi_1, \psi_2 \in \DQBF{V}$ we denote DQBF formula in which every occurence of $\psi_1$ in $\psi$ is replaced by $\psi_2$ except for the occurence of $\psi_1$ in the dependency sets. If both $\psi_1, \psi_2 \in V$, then $\psi_1$ is also replaced by $\psi_2$ even in the dependency sets of $\psi$.

%dat definiciu semantiky
Definition of satisfiability of DQBF in NNF follows the same way as in previous section but now we have to decide what to do with free variables. In non-prenex case of QBFs, free variables are usually assumed to be existential variables with quantifiers at the beginning of the formula. In DQBF case this would mean that free variables have empty dependency sets so the definition needs to take care of that. We first define Skolem function candidates which are possible candidates fulfilling conditions set by dependency sets. Then we define \emph{semantics} of DQBFs in NNF as the set of Skolem function candidates which satisfy the formula. % follows this reasoning???? To have easier time with proofs we define semantics of DQBFs as a set of Skolem functions satisfying the formula.??? definition of semantics???? is a bit more convoluted. We first define Skolem function candidates from which we choose the set of 



\begin{definition}[Skolem Function Candidates]
  Let $\psi \in \DQBF{V}$. We say that function $s\colon (\evars{\psi} \cup \fvars{\psi}) \to \BFuncs{\uvars{\psi}}$ is a \emph{Skolem function candidate} of $\psi$ if
  \begin{itemize}
      \item $\supp{s(v)} \subseteq (D_v \cap \fvars{\psi})$ for all $v \in \evars{\psi}$ and
      \item $\supp{s(v)} = \emptyset$ for all $v \in \fvars{\psi}$, i.e. $s(v)$ is either represented by $0$ or $1$.
  \end{itemize}
\end{definition}

The set of all Skolem function candidates of $\psi$ is denoted by $\Scands{\psi}$. If $s \in \Scands{\psi}$, we write $s(\psi)$ for the formula that results from $\psi$ by replacing each existential and free variable $v$ by a BF that represents $s(v)$ and omitting all quantifiers from $\psi$. This results in a BF containing only variables from $\uvars{\psi}$.

\begin{definition}[Semantics]
  Let $\psi \in \DQBF{V}$. We define the \emph{semantics} $\seman{\psi}$ of $\psi$ as follows:
  \[\seman{\psi} = \set{s \in \Scands{\psi} | v(s(\psi))=1 \text{ for all } v \in \valtns{\uvars{\psi}}}.\]
\end{definition}
Formula $\psi$ is \emph{satisfiable} if $\seman{\psi} \not= \emptyset$. The elements of $\seman{\psi}$ are called \emph{Skolem functions} of $\psi$. We say that two formulas $\psi_1$ and $\psi_2$ are \emph{equisatisfiable}, denoted $\psi_1 \approx \psi_2$, if both are either satisfiable or both are unsatisfiable.
\begin{example}
\label{ex:DQBF:NNF}
To explain semantics of DQBF in NNF we use an example from~\cite{HQSquantifierLocalization}:
\[\psi = \forall x_1 \forall x_2 : (x_1 \lequal x_2) \lor \exists y (x_2) : \neg(x_1 \lequal y)\]
where ${\lequal}$ has the same meaning as in Example~\ref{ex:BF:prenex}. The meaning behind this formula is that either $x_1$ and $x_2$ are the same or there is some $y$ which is dependent only on $x_2$ which is different from $x_1$. The Skolem function candidates of $\psi$ are then $\Scands{\psi} = \{y\mapsto f_0$, $y\mapsto f_1$, $y\mapsto f_x$, $y\mapsto f_{\neg x}\}$ % then map $y$ to Boolean function which are represented either by $0,1,x_2$ or $\neg x_2$. It is pretty obvious that only 
where $f_0$ is Boolean function represented by Boolean formula $0$, $f_1$ is represented by $1$, $f_x$ by $x$ and $f_{\neg x}$ by $\neg x$. It is pretty obvious that only $s = y \mapsto f_{\neg x}$ is a Skolem function of $\psi$, since only $s(\psi) = (x_1 \lequal x_2) \lor \neg(x_1 \lequal \neg x_2)$ is true in all valuations.
\end{example}

\subsection{Localising Quantifiers}
Now that we have defined DQBFs we give some theoretical results which show the equisatisfiability of different DQBFs which can result in their simplification. These results can be then used for developing a DQBF solver which checks whether an input DQBF is satisfiable or not.

We start with two theorems that show when it is possible to push (extract) quantifiers in (from) the formula. %The first theorem is from~\cite[Theorem 3]{HQSquantifierLocalization} where ... is replaced by ... from~\cite[Theorem 4]{HQSquantifierLocalization}

\begin{theorem}[{\cite[Theorem 3,4]{HQSquantifierLocalization}}]
  Let ${\diamond} \in \set{{\land},{\lor}}$ and $\psi,\psi_1,\psi_2 \in \DQBF{V}$. We assume that $x'$ and $y'$ are fresh variables, which do not occur in $\psi,\psi_1$ and $\psi_2$. The following equisatifiabilities hold for all DQBFs in NNF:
  \begin{align}
      %\forall x : \psi &\approx \psi \text{, if } x \not\in V_{\psi} \\
      \forall x : (\psi_1 \land \psi_2) &\approx (\forall x : \psi_1) \land (\forall x' : \psi_2\substitute{x}{x'}) \\
      \forall x : (\psi_1 \land \psi_2) &\approx (\psi_1^{-x} \land (\forall x : \psi_2)) \text{, if } x \not\in V_{\psi_1} \\
      \forall x : (\psi_1 \lor \psi_2) &\approx \begin{multlined}[t][3cm](\psi_1 \lor (\forall x : \psi_2)) \text{, if } x \not\in V_{\psi_1}\\
      \text{ and } x \not\in D_y \text{ for all } y \in \evars{\psi_1}\end{multlined} \\
      \exists y(D_y) : (\psi_1 \lor \psi_2) &\approx (\exists y(D_y) : \psi_1) \lor (\exists y'(D_y) : \psi_2\substitute{y}{y'}) \\
      \exists y(D_y) : (\psi_1 \diamond \psi_2) &\approx (\psi_1 \diamond (\exists y(D_y) : \psi_2)) \text{, if } y \not\in V_{\psi_1} \\
      \exists y_1(D_{y_1})\exists y_2(D_{y_2}) : \psi &\approx \exists y_2(D_{y_2}) \exists y_1(D_{y_1}) : \psi \\
      \forall x_1\forall x_2 : \psi &\approx \forall x_2\forall x_1 : \psi \\
      \forall x\exists y(D_y) : \psi &\approx \exists y(D_y)\forall x : \psi\text{, if } x \not\in D_y.
  \end{align}
\end{theorem}

The next theorem shows that if DQBFs from the previous theorem occur as subformulas of DQBF $\psi$ they can be replaced without changing the satisfiability of $\psi$.

\begin{theorem}
  Let $\psi \in \DQBF{V}$ and $\psi_1, \psi_2$ be the formulas from previous theorem on each side of the same equality where $\psi_1$ is a subformula of $\psi$. Again, we assume that $x'$ and $y'$ are fresh variables which do not occur in $\psi$. Then $\psi \approx \psi'$ where $\psi'$ results from $\psi$ by replacing $\psi_1$ with $\psi_2$.
\end{theorem}
\begin{proof}
  TODO!!!
\end{proof}

\subsection{Quantifier Elimination}
The next two theorems give a possibility to eliminate both universal and existential quantifiers. Universal quantifiers can be eliminated always but by doing it the resulting DQBF can contain new copies of existential variables.

\begin{theorem}[Universal Expansion]
  Let $\psi, \psi_1 \in \DQBF{V}$ such that $\forall x : \psi_1$ is a subformula of $\psi$. Let $E_{x} = \set{y \in \evars{\psi_1} | x \in D_y}$ be the set of all existential variables dependent on $x$. Let $\psi_2$ be a DQBF that results from $\psi_1$ by substituting each $y \in E_x$ with a new variable $y' \not\in V_{\psi}$ where we set $D_{y'} = D_y$. Then $\psi \approx \psi'$ where $\psi'$ results from $\psi$ by replacing the subformula $\forall x : \psi_1$ by 
  \[\xi = \psi_1^{-x}\substitute{x}{0} \land \psi_2^{-x}\substitute{x}{1}.\]
\end{theorem}
\begin{example}
Let $\psi$ be DQBF from Example~\ref{ex:DQBF:NNF}. Using universal expansion we can transform it to
\begin{multline*}
    \psi' = \forall x_1 : ((x_1 \lequal 0) \lor \exists y (\emptyset) : \neg(x_1 \lequal y)) \land\\
    ((x_1 \lequal 1) \lor \exists y' (\emptyset) : \neg(x_1 \lequal y'))
\end{multline*}
which (like $\psi$) is satisfiable.
\end{example}
\begin{proof}
  First notice that each $y \in E_x$ must be entirely inside $\psi_1$ (even the ''$\exists y$'' bit). This stems from the way the set $\DQBF{V}$ is constructed; the dependency sets of new existential variables can only be formed from free or completely new variables and if ''$\exists y$'' was outside $\forall x :\psi_1$, then $x \in D_y$ would have to be a free or new variable. With that we can now prove that $\seman{\psi} \not= \emptyset$ iff $\seman{\psi'} \not= \emptyset$.
  
  First we prove that if $\seman{\psi} \not= \emptyset$, then $\seman{\psi'} \not= \emptyset$. Let $s \in \seman{\psi}$. We build a Skolem function $s'$ for $\psi'$. For all $y \in (\evars{\psi} \setminus E_x)$ set $s'(y) = s(y)$. For $y \in E_x$ and its copy $y' \in \evars{\psi_2}$ we set $s'(y) = s_{x=0}(y)$ and $s'(y') = s_{x=1}(y)$ where $s_{x=0}(y)$ and $s_{x=1}(y)$ denote Boolean functions that behave like $s(y)$ where $x$ is set to $0$ or $1$ respectively. Then $s'$ is a Skolem function candidate of $\psi'$. We now have to prove that BF $s'(\psi')$ is true in all valuations over $\uvars{\psi'}$. Notice that $\uvars{\psi'} = \uvars{\psi} \setminus \set{x}$. Let $v$ be one such valuation and denote $v_{x\mapsto0}$ and $v_{x\mapsto1}$ valuations over $\uvars{\psi}$ such that $v(x') = v_{x\mapsto0}(x') = v_{x\mapsto1}(x')$ for $x' \in \uvars{\psi'}$, $v_{x\mapsto0}(x) = 0$ and $v_{x\mapsto1}(x) = 1$. Because $s \in \seman{\psi}$, it has to hold that $v_{x\mapsto0}(s(\psi)) = v_{x\mapsto1}(s(\psi)) = 1$. Also, the only difference in $s(\psi)$ and $s(\psi')$ is in $s(\forall x:\psi_1)$ and $s'(\xi)$. Therefore, if $v_{x\mapsto0}(s(\forall x:\psi_1)) = v_{x\mapsto1}(s(\forall x:\psi_1))$, then $v(s'(\xi))$ is equal to that and so $v(s'(\psi')) = 1$. If $v_{x\mapsto0}(s(\forall x:\psi_1)) \not= v_{x\mapsto1}(s(\forall x:\psi_1))$ then $v(s'(\xi))$ is equal to one of them but because in both $v_{x\mapsto0}$ and $v_{x\mapsto1}$ the formula $s(\psi)$ evaluates to 1 it has to hold that $s'(\psi')$ evaluates to 1 in $v$.
  
  Now we prove the other direction. Let $s' \in \seman{\psi'}$. We build a Skolem function $s$ for $\psi$. For all $y \in (\evars{\psi} \setminus E_x)$ we set $s(y) = s'(y)$. For $y \in E_x$ with a copy $y' \in \evars{\psi_2}$ we set $s(y)$ to Boolean function that is represented by BF $(\neg x \land s'(y)) \lor (x \land s'(y'))$ where $s'(y)$ and $s'(y')$ are replaced with BFs representing them. It is obviously a Skolem function candidate thus we need to prove that $s(\psi)$ is true in all valuations. Let $v$ be a valuation over $\uvars{\psi}$. Without loss of generality assume that $v(x) = 0$. Then $v(s(y)) = (\neg 0 \land s'(y)) \lor (0 \land s'(y')) = s'(y)$. This means that $v(s(\forall x : \psi_1)) = v(s(\psi_1^{-x}\substitute{x}{0}))$. Therefore if $v(s'(\xi)) = 1$ then $v(s(\forall x : \psi_1)) = 1$. Because $\psi'$ is in NNF, $s'(\xi)$ occurs in $s'(\psi')$ in a tree of ${\land}$ and ${\lor}$ operations. However, from assumption we know that $v(s'(\psi')) = 1$. Therefore if $v(s'(\xi)) = 0$ it does not matter, $v(s'(\psi'))$ would be true even if $v(s'(\xi)) = 1$. Also $s(\psi)$ and $s'(\psi')$ differs only in $s(\forall x : \psi_1)$ and $s'(\xi)$ part, therefore $v(s(\psi))$ must be equal to 1. 
\end{proof}

% TODO: check if we need this:
%In the case that $x \not\in V_{\psi_1}$ the situation is simpler --- we just remove the quantifier:
%\begin{lemma}
%Let $\psi, \psi_1 \in \DQBF{V}$ such that $\forall x : \psi_1$ is a subformula of $\psi$ and $x \not\in V_{\psi_1}$. Then $\psi \approx \psi'$ where $\psi'$ results from $\psi$ by replacing the subformula $\forall x : \psi_1$ by $\psi_1$.
%\end{lemma}
%\begin{proof}
%  For $\psi$ to be a DQBF in NNF, it has to hold that $x$ is not quantified anywhere else than in the subformula $\forall x : \psi_1$. Then the equisatisfiability follows from the similar argument as was used in the proof of $\forall x : \psi_1 \approx \psi_1$ in~\cite[Theorem 3]{HQSquantifierLocalization}.
%\end{proof}

The conditions when existential elimination is possible are more convoluted:

\begin{theorem}[{\cite[Theorem 5]{HQSquantifierLocalization}}]
  Let $\psi, \psi_1 \in \DQBF{V}$ such that $\exists y(D_y) : \psi_1$ is a subformula of $\psi$ where $\psi_1$ does not include any quantification and includes only variables from $D_y \cap V_{\psi}^{\text{free}} \cap \set{y' \in V_{\psi}^{\exists} | D_y' \subseteq D_y}$. Then $\psi \approx \psi'$ where $\psi'$ results from $\psi$ by replacing the subformula $\exists y(D_y) : \psi_1$ by $\psi_1\substitute{y}{0} \lor \psi_1\substitute{y}{1}$.
\end{theorem}
\begin{example}
Let
\begin{multline*}
\psi = \forall x_1 \forall x_2 \exists y_1(x_1) : ((x_1 \land x_2) \lequal y_1)\\
\lor \exists y_2 (x_1,x_2) : (x_1 \lequal y_2) \land (y_1 \lequal y_2)
\end{multline*}
be a DQBF that says that either $y_1$ has the same value as $x_1 \land x_2$ (which is impossible because $y_1$ depends only on $x_1$) or there is some $y_2$ which is equal to both $y_1$ and $x_1$ (which is possible because both $y_1$ and $y_2$ depend on $x_1$). This means that this formula is satisfiable. Because the subformula $\exists y_2 (x_1,x_2) : (x_1 \lequal y_2) \land (y_1 \lequal y_2)$ fulfills the conditions from the theorem, $\psi$ can be transformed to
\begin{multline*}
\psi = \forall x_1 \forall x_2 \exists y_1(x_1) : ((x_1 \land x_2) \lequal y_1)\\
\lor (((x_1 \lequal 0) \land (y_1 \lequal 0)) \lor ((x_1 \lequal 1) \land (y_1 \lequal 1)))
\end{multline*}
which is still satisfiable.
\end{example}


%dependency elimination - jedine ak to budem pouzivat, treba potom aj dokaz ze to funguje pre podformule - inak je to spomenute pri state of art

%veci k zamysleniu - zrobit vseobecny dokaz alebo pre kazdu vec zvlast + ake priklady dat, asi dat priklad ku kazdej vete??


% \begin{theorem}
%   Let $\phi, \psi_1, \psi_2$ be a ??non-prenex?? DQBFs where $\psi_1$ is subformula of $\phi$ and $\psi_1 = \psi_2$. Assume that for each existential variable $y \in V_{\psi_1}^{\exists}$ the dependency set of $y$ in $\psi_1$ is equal to the dependency set of $y$ in $\psi_2$. Then it holds that $\phi = \phi\substitute{\psi_1}{\psi_2}$. 
% \end{theorem}
% \begin{proof}
%   We show that if $s \in \seman{\phi}$ then $s \in \seman{\phi\substitute{\psi_1}{\psi_2}}$ and vice versa. Notice that $V_{\phi}^{\forall} = V_{\phi\substitute{\psi_1}{\psi_2}}^{\forall}$ and $V_{\psi_1}^{\forall} = V_{\psi_2}^{\forall}$. Let $s \in \seman{\phi}$ and assume that $s \not\in \seman{\phi\substitute{\psi_1}{\psi_2}}$.  That means there exists a valuation $v\colon V_{\phi}^{\forall} \to \set{0,1}$ such that $v(s(\phi)) = 1$ but $v(s(\phi\substitute{\psi_1}{\psi_2})) = 0$. Because these two formulas are the same except for the difference in $\psi_1$ and $\psi_2$ the change of evaluation has to come from the difference in evaluation of these two formulas. That is $v(s(\psi_1)) \not= v(s(\psi_2))$. If $\psi_1 = v$ for some $v \in V$, then $\psi_2 = v$ and then $v(s(\psi_1)) = v(s(\psi_2))$. This means both $\psi_1$ and $\psi_2$ are something more complex than just a variable which cannot have negation in front of it in $\phi$ (because negation can be only in front of a variable). This also means that BF $s(\phi)$ will have $s(\psi_1)$ only in sequence of conjunctions and disjunctions. But because $v(s(\phi))=1$ and $v(s(\phi\substitute{\psi_1}{\psi_2}) = 0$ where the only difference in these formulas are $s(\psi_1)$ and $s(\psi_2)$ it has to hold that $v(s(\psi_1)) = 1$ and $v(s(\psi_2)) = 0$. Let $s'$ be a restriction of $s$ on $V_{\psi_1}^{\exists} \cup V_{\psi_1}^{free}$. If $s' \in \seman{\psi_1}$ then from $\psi_1 = \psi_2$ it has to hold that $s' \in \seman{\psi_2}$ and so $v(s(\psi_2)) = 1$. This means that $s' \not\in \seman{\psi_1}$.
% \end{proof}


\section{Binary Decision Diagrams}
%Having defined DQBFs we can now turn our attention to their representation. We have already defined some normal forms in section blabla, but algorithmically, working with them in this form would not be very effective (why?). For this reason binary decision diagrams (BDDs) were introduced~\cite{BDD}. These structures allow for a succinct representation of BFs with an efficient polynomial time implementation of logical operations.
In this section we give definition of \emph{binary decision diagrams}~\cite{BDD} which are a succinct representation of BFs with an efficient polynomial time implementation of logical operations.

\begin{definition}
  Let $V$ be a set of variables. A \emph{binary decision diagram} (BDD) over $V$ is a binary tree like structure with two terminal nodes 0 and 1 and a set $N$ of non-terminal nodes each representing a variable from $V$. Each non-terminal node has two child nodes with edges connecting them labeled with 0 and 1 respectively.
\end{definition}

%We will give a definition of reduced ordered binary decision diagram, however we do not use that name and whenever we say BDD we mean ROBDD.

Let $v\colon V \to \{0,1\}$ be a valuation over $V$. BDD $\beta$ in this valuation is evaluated to 0 or 1 by moving along the edges of non-terminal nodes, where if we are in the node representing variable $x \in V$, then we move along the edge labelled by $v(x)$. The resulting terminal node is then the evaluation of $B$ in $v$. BDD $\beta$ then represents a Boolean formula $\phi \in \BF{V}$ if for each valuation $v$ it holds that $\beta$ is evaluated to $v(\phi)$.

dat tu figure s bdd a robdd ktore reprezentuju rovnaku funkciu

\begin{example}
Fig. ??? shows an example of a BDD for BF ... Edges labelled by 0 are dashed???, by 1 are full??? 
\end{example}

To have simpler algorithms for BDDs, we actually use \emph{reduced ordered BDDs} (ROBDDs) which are BDDs such that 
\begin{enumerate}
    \item the variables on each path from the root of BDD to the terminal are in the same order,
    \item the isomorphic (''the same looking'') subtrees are merged and
    \item nodes whose both edges go to the same subtree are deleted.
\end{enumerate}
They have the nice property that for each BF there is exactly one ROBDD representing it. In the following, when we use BDD we mean ROBDD. see fig somehting for something

%To have simpler algorithms for BDDs, we usually want to have some ordering on the set of variables $V$ such that for each path from the root of BDD to the terminal of the BDD has nodes in this order. These BDDs are called ordered BDDs (OBDDs). The ordering of BDDs is important, because two BDD representations of the same BF in different orderings can have very different sizes. odkazat sa na obrazok ze tam je ordered 

%Two OBDDs with the same ordering representing the same BF can still be different, that is why we will actually work with reduced OBDDs (ROBDDs) which is an OBDD that merges isomorphic (''the same looking'') subtrees and deletes nodes whose both edges go to the same subtree. 

%As a final step 

% chceme mat linearne operacie ---> treba aby boli usporiadane ---> shannon expansion ---> ukazat operacie???
%When we have two BDDs, we want to have an effective way to compute their conjunction/disjunction. Such an algorithm should just traverse the BDDs in the same way while comparing the variables in the nodes. However, variables in the nodes on the paths from the root in general BDDs can occur in any order, which would cause problems with simple traversal of BDDs. To solve this we add an ordering on the set of variables $V$. The nodes in the BDD are then sorted based on this ordering. The resulting BDD is called an ordered binary decision diagram (OBDD).

%\begin{example}
%ukazat asi z previous example 2 orderings a povedat daco o tom ze operacie sa daju teraz implementovat jednoducho len prechadzanim stromom
%\end{example}

%nieco o ROBDD, ze pre kazde BF existuje prave jedno unikatne pre kazde zoradenie + ze vyber zoradenia je dolezity + ze da sa dynamicky preukladat???
%OBDDs still suffer from one problem -- for each BF we would like to have for given order on $V$ only one OBDD representing it. This is not the case, notice OBDDs on fig. ????? and ???. Both represents BF ??? but they are different. That is why we introduce reduced OBDDs (ROBDD) (vysvetlit lepsie ze to neni moj prinos). ROBDD is an OBDD such that all isomorphic sub-BDDs are merged and the nodes that have the same BDD as children are removed.

%\begin{example}
%porovnat nejake OBDD s jeho ROBDD, detailne (kde sa ukaze )
%\end{example}

For each operation on BFs (${\land}$, ${\lor}$, ${\neg}$) there is an algorithm on BDDs which can get the resulting BDD representing the application of the operation. For example, if we have BDD representing BF $\psi_1$ we can get BDD representing $\neg \psi_1$ by swapping terminal nodes $0$ and $1$. See for example~\cite{BDDdetails} for more details. %If we have another BDD representing BF $\phi_2$, we can get $\psi_1 \land \psi_2$ by using this algorithm.  Similarly an algorithm for ${\lor}$ can be defined.

%In the following we will use the term BDD instead of ROBDD.

%\section{And-Inverter Graphs - je ich treba rozobrat???}

\chapter{State of the Art}
%problem - vytvorit algoritmus ktory zisti ze ci dana DQBF formula je splnitelna alebo nie pomocou BDDcok alebo jej podobnym strukturam (mozno dat do teorie)

%popisat tu aj nejak historiu DQBF, akoze napriklad ten clanok s henkins qunatifiers abo daco???

%Having laid out all the necessary theory we can now define the problem we are trying to solve and current implementations solving it.

%\section{The Problem of Satisfiability}
%Let $\psi$ be DQBF. The problem we are trying to solve is to tell whether this formula is satisfiable, that is if there exists model in which it is true. For BF, this problem belongs to NP-complete class of problems cite. If we move to QBF, the problem gets harder and belongs to PSPACE-complete class. As was shown in CITE, this problem for DQBF belongs in NEXPTIME-complete class of problems, that is it is even harder.

%\section{State of the Art}
%ake solvre existuju (vybrat tie s ktorymi budem porovnavat?), trocha viac popisat HQS a ze ako funguje aj HQSpre (lebo ho pouzivam) - ale fakt len zopar veci spomenut, mozno aj to ze mozme pouzit vytvaranie log. gates pretoze to menim do BDD ako to robia v HQS
In this chapter we give an overview of existing solutions and solvers which tackle the satisfiability problem of DQBF. An overview of existing solvers can also be found in~\cite{DQBFStateOfArtTalk,DQBFStateOfArt,DQBFStateOfArtNew}. We give a short overview of most solvers and techniques but % that is why we just give a short updated overview here %(short for DQDPLL, iDQ, iProver and dCAQE). 
%However 
we go into more details for solver HQS, because the methods we use for developing our solver are based on the workings of this solver.
%current solutions, that already exist

%nieco o tom ze vacsina funguje na CNF, ale HQS a moje bude na NNF

%fast DQBF refutation - ze premienaju na QBF abo daco (bunsat)

\section{First Solution - DQDPLL}
The first solver that tackled the satisfiability problem for DQBF was based on DPLL(TODO cite) algorithm which is successfully used for BF and QBF solvers(TODO cite??). This algorithm works on formulas in CNF by searching for a satisfying assignment based on the clauses in CNF. By recursively choosing an assignment for literals and checking whether CNF stay consistent, DPLL searches trough all assignments until a satisfying one is found or all of them are decided to be unsatisfying. An adaptation called DQDPLL~\cite{DPLLalgorithm} was introduced for DQBF which extended existing solutions for QBF by adding so-called Skolem clauses which encode the dependencies between existential and universal variables. However, by doing so the algorithm becomes too slow and results in noncompetitive solver.

\section{Instantiation -- iDQ and iProver}
The first efficient DQBF solver, called iDQ~\cite{iDQandDQDIMACS}, is based on instantiation technique used for solving Effectively Propositional Logic (EPR)~\cite{iProver}. This solver also works only on formulas in CNF where in each step of the algorithm it tries to create a BF $\phi$ that is an overapproximation of an input DQBF. This is done by instantiating some set of clauses, that is it applies universal expansion locally to them. This BF $\phi$ is then checked for satisfiability, where if $\phi$ is unsatisfiable then the input DQBF is also not satisfiable, while if it is satisfiable, it must be checked if the resulting valuation is valid for the input DQBF. If it is not, then it is used to create more clause instances which are then used to refine this overapproximation.

However, a solver for EPR can also be used directly for DQBF. Because EPR belongs to the same complexity class as DQBF, there exists a polynomial-time reduction from DQBF to EPR~\cite{iDQandDQDIMACS}. This is used by EPR solver iProver~\cite{iProver} which transforms DQBF in CNF to EPR instance and then solves this. In the SAT competition??? it turned out that iProver is more powerful compared to iDQ??

\section{Clausal Abstraction -- dCAQE}
Another DQBF solver called dCAQE~\cite{dCAQE} is based on clausal abstraction (mozno citovat). This solver\,---\,working again on DQBF in CNF\,---\,first puts universally and existentially quantified variables to some sets (called nodes) which are then divided into levels based on the ordering of dependencies of existential variables. The algorithm then construct for each node on each level a BF that represents which clauses in CNF it can satisfy (for existential node) or falsify (for universal node). The algorithm then builds a candidate valuation by processing each level. For each level either this valuation is extended, there is some conflict which means that the algorithm has to backtrack to some lower level and refine the abstraction or the candidate valuation is a satisfying one ending the algorithm.

%odkaz na sekciu o PSPACE sublassse, ze je mozne ze tento solver nie je uplny

%mozno spomenut fork splitting or some shit like that, v arXiv rozsirenej verzii maju nieco o tom v appendixe

\section{Quantifier Elimination -- HQS}
The next solver is based on quantifier elimination. This solver's basic premise is simple\,---\,it iteratively chooses some universal variable for universal expansion (theorem ???) thus eliminating it and then eliminates all existential quantifiers that are dependent on all leftover universal quantifiers using theroem???.
%nejak vysvetlit ze pouzivaju AIG

\begin{algorithm}
\caption{Quantifier elimination algorithm}
\label{alg:QE}
\begin{algorithmic}[1]
\Function{SolveDQBF}{DQBF $\psi$ in prenex form}
\While{$V_{\psi}^{\forall}$ is not empty}%there is some universal variable $x$ in $\psi$
    \State choose $x$ from $V_{\psi}^{\forall}$
    \State $\psi = \forall\text{-elimination}(x,\psi)$% universal expansion
    \ForAll{$y \in V_{\psi}^{\exists}$ s. t. $D_y = V_{\psi}^{\forall}$} %existential variables $y$ such that they depend on all universal variables in $\psi$
        \State $\psi = \exists\text{-elimination}(y,\psi)$% existential expansion
    \EndFor
\EndWhile
\EndFunction
\end{algorithmic}
\end{algorithm}

In~\cite{HQSsimpleAlg} the authors introduced the basic algorithm for solving DQBF in this way. %(mozno spomenut ze to vlastne vymysleli aj v iDQ clanku, ako nejaky base solver ktory pouziva BDDcka) 
Algorithm~\ref{alg:QE} shows simple pseudocode of this. In every step this algorithm chooses (using some heuristic) an universal variable to eliminate. In~\cite{HQSsimpleAlg} they used a heuristic where the universal variable is chosen based on the number of existential ones that depend on it --- they choose the one that has minimal number of dependencies. After that, all the existential variables that depend on everything are also eliminated. To use this algorithm it is important to choose a good representation of the matrix of the DQBF. For this the authors chose and-inverter graphs (AIG) which are a tree-like structures where non-terminal nodes represent conjunction, terminal nodes are variables and these are connected with two types of nodes: normal ones and those that represent negation. With a proper implementation of operations they can be optimised in such a way that they reuse parts of the graphs resulting in functionally reduced AIGs~\cite{FRAIGs}.
%mozno aj tu spomenut tie theorems

%MYSLIENKA K TODO - heuristika ze vyberam podla poctu zavislosti v originalnom DQBF alebo v tom co sa mi postupne po odstranovani vytvara

% bol zlepseny s tym ze pridali nejaky preprocessing, odstranovanie pure a daco vars a QBF solver ako subproceduru - dat pekny obrazok
Following this technique,  the authors of~\cite{HQSquantifierElimination} introduced solver HQS which was enhanced by using QBF solver as a subprocedure. They still eliminate quantifiers in similar fashion, but now they do it until the formula can be transformed to QBF. On this QBF they then run already existing solver for QBF called AIGSolve~\cite{AIGSolve} which can use more effective techniques developed for QBFs. Universal quantifiers to eliminate are chosen in the beginning in such a way that the number of universal eliminations is as small as possible while still the resulting formula is QBF. For this they build a dependency graph in which nodes are existential variables where $y_i$ is connected to $y_j$ if for their dependency sets hold that $D_{y_i} \not\subseteq D_{y_j}$. They noticed that if this graph is acyclic, the formula can be seen as QBF. Furthermore, there is a cycle in this graph iff there is a simple cycle in the graph (between two nodes). They use this to create an instance of MaxSAT problem which is a problem of finding a valuation of one BF while maximizing another formula. This instance then encodes which universal variables have to be eliminated so the dependency graph becomes acyclic while minimising the number of universal variables to eliminate. After finding the set of variables to eliminate, they follow with an improved quantifier elimination algorithm. They improved it by adding a preprocessing step (see section~\ref{sec:preprocessing}), syntactic detection and elimination of special types of variables called unit and pure and already mentioned subprocedure that calls QBF solver AIGSolve.

In~\cite{HQSdependencyElimination} the algorithm was changed a bit again. The authors were interested in whether it is possible to somehow remove specific dependencies from the list of dependencies of existential variable. They have shown that it is possible, albeit at the expense of adding a new existential variable. This is similar to the already shown universal expansion (theroem???), however, by removing just one dependency we add just one new existential variable. Let for example
\[\psi = \forall x_1 \dots \forall x_n \exists y_1(D_{y_1}) \dots \exists y_m(D_{y_m}) \colon \phi\]
be a prenex DQBF where $\phi$ is a BF and $x_1 \in D_{y_1}$. We can now remove $x_1$ from $D_{y_1}$ by adding a copy of $y_1$ resulting in an equisatisfiable DQBF
\begin{multline*}
\forall x_1 \dots \forall x_n \exists y_1^0(D_{y_1} \setminus \{x_1\}) \exists y_1^1(D_{y_1} \setminus \{x_1\}) \exists y_2(D_{y_2})\dots \exists y_m(D_{y_m}) \colon \\
\phi\left[((\neg x_1 \land y_1^0) \lor (\neg x_1 \land y_1^1))/y_1\right].
\end{multline*}
Using this the authors do not look for a set of universal variables to eliminate, but for partial dependencies whose removal will result in a DQBF which can be transformed into QBF. Again, they create a dependency graph which is acyclic iff DQBF can be seen as QBF. But now they create a bipartite graph whose two sets of nodes are the set of universal variables and the set of existential variables which are connected based on whether universal variable is in the dependency set of existential one. Similarly to previous, they use this graph to solve an optimization problem that encodes which dependencies to remove to get QBF which can then be solved with AIGSolve.
%This was good and all, but in~\cite{HQSdependencyElimination} they zaoberat sa whether it is possible to remove only some universal variables from dependency sets. They showed it is possible ....tu asi dat cely ten teorem ak som ho nedal do teorie...

Last but not least, in~\cite{HQSquantifierLocalization} the authors have given a theoretical foundations for non-prenex DQBFs and the possibility to push and eliminate quantifiers inside formulas as explained in section ????. Using this they improve HQS by adding another step in the algorithm, where they first push quantifiers as much as possible into the formula, eliminate the existential variables using theorem ??? and then return the formula back to the prenex form. After this they start with elimination of universal quantifiers as was explained in previous paragraphs.

\section{Preprocessing}
\label{sec:preprocessing}
In previous sections we give an overview of techniques used for implementations of different solvers. This sections explains some techniques that are used before the actual solvers run, called fittingly preprocessing.


%We have just gave an overiew of different solvers for DQBF. However, 
%preprocessing pre QBF je dolezity, ergo asi je aj pre DQBF a teda je vela vyzkumu na preprocessing techniques

\subsection{HQSpre}
The first and only preprocessing engine for DQBF called HQSpre~\cite{HQSpre1,HQSpre2,HQSpreJournal} took many techniques used for QBF preprocessing and lifted them to the DQBF case. We go into details only for one technique called gate extraction. As was explained in Section~\ref{sec:BF:CNF}, transforming BF to CNF can result in polynomial increase of the size of the formula by adding new variables. Gate extraction works in opposite direction\,---\,it tries to extract this newly introduced variables and turn the formula back into one that is not in CNF. This does not mean that gate extraction works only on formulas in CNF, it can be used generally by finding ...connections????..... which makes it possible to replace a variable by some Boolean function of existing variables. However, this technique can only be used by solvers which can work on DQBFs which are not in CNF. Currently, that is only HQS. For a thorough explanation of other techniques used by HQSpre we refer the reader to~\cite{HQSpreJournal}. 

\subsection{Approximations}
Another technique which is used in some solvers as preprocessing step is to find some BF or QBF approximations which can be used with existing solvers for BFs/QBFs to potentially give us faster result about the input DQBF (satisfiability for underapproximations, unsatisfiability for overapproximations). In~\cite{HQSsimpleAlg} QBF approximations are introduced which are created by changing the dependencies of existential variables that result in QBF with the same matrix. However, these approximations can be precise only up to some level. By adding some information to these QBF overapproximation (thus changing the matrix), the authors in~\cite{FastDQBFRefutation} were able to create more and more precise QBF overapproximations. For more thorough explanation of these approximations see~\cite[Section~3]{HQSpreJournal}.

\subsection{PSPACE-complete subclass}
The last preprocessing technique is simply put just a check whether DQBF does not belong to a subclass of problems which are PSPACE-complete and if it does, it is transformed to QBF on which existing QBF solver is run. The simplest such subclass would be those DQBFs whose dependencies are linearly ordered. Then it can be easily transformed into QBF by reordering the quantifiers. In~\cite{PSPACEsubclass} the authors investigated whether there is some another DQBF subclass which is PSPACE-complete. They show that DQBFs where the dependency sets are either equal or pairwise disjoint can be transformed into QBFs with just linear increase in size. Thus having a check on whether an input DQBF belongs to this class before running a DQBF solver and if it does transforming it into QBF and running a QBF solver can improve the efficiency.

%As a note, 

%If we can find a subclass that is pspace we can translate to QBF - obviously for DQBF that is already in linear order can be seen as BQF
%QBF can be viewed as a DQBF with linearly ordered dependencies. This is enough to make it PSPACE-complete and so, in PSPACECLANOK they were interested in whether there is some another sublass of DQBF which is PSPACE-complete. They have shown that  
%existuje classa DQBFs ktore idu iba s polynomialnym zvacsenim previest na QBF ---> patri do PSPACE a je mozne to pouzit ako preprocessing step kde ak zistime ze sme v tejto classe tak pustime QBF solver dajaky
%As a note toto vzislo z toho, ze clanok zrobil dajaky system ktory je rozsirenie QBF systemu kde pridali fork extension pravidlo. They showed that this system is complete and sound for DQBF, however, in dCAQEclanok (ale v appendixe extended version) and PSPACEclanok ukazali ze to tak nie je (nie je complete ci co). Obe tieto clanky ukazali ze tento system je complete pre tuto PSPACE classu. Zaroven dCAQE vyuziva tento fork extension ???a zle si myslia ze tato classa je NEXPTIME??? vo svojom algoritme, takze je moznost ze ich algoritmus nefunguje pre celu classu DQBF


% vysvetlit prve dve clanky, ze ide o jednoducho univ. expansion -> exist. elimination pouzitim AIG a pridanim nejakych zaujimavych drobnosti sa vytvori QBF ktory sa hodi do nejakeho existujuceho solvru (pridat obrazok toho cyklu + naivny alg.), tu spomenut aj preprocessor ze je vysvetleny niekde
% potom vysvetlit ze je dolezity spravny vyber univ. kv. na expansion a vysvetlit ze v naivnom alg. to je jednoduche proste tie na ktore zavisia najviac, v tych dalsich clankoch vytvoreny dependency graf ktory sa nejak riesi, neskor sa odstranuje priamo jedna zavislost pre exist. kvantifikatory
% potom zacat hovorit o tom ze ako pchat kvantifikatory do vnutra, kde to nejak zrobili
% vysvetlit preprocessor (hlavne gate reduction alebo jak to nazvat)
% spomenut asi aj overapproximations

\chapter{Algorithm}
\label{chap:algorithm}

\section{High level defintion???}

\section{Heuristics of deciding what to do on each level of quantifier tree}
\begin{itemize}
    \item nothing - simple solver
    \item remove everything possible (all universal vars and possible exist. vars) - tree solver
    \item remove only exist vars (or maybe just some univ. vars - those that only few exist vars depend on or those that will allow me to remove some exist vars)???
    \item remove some dependencies to be able to remove exist vars????
\end{itemize}

\section{Heuristics for order of elimination of universal variables}

\chapter{Implementation}

\chapter{Experimental results}
grafygrafygrafy

\chapter*{Conclusion}
\addcontentsline{toc}{chapter}{Conclusion}

\printbibliography[heading=bibintoc]

%\chapter{Inserting the index}
%After using the \verb"\makeindex" macro and loading the
%\texttt{makeidx} package that provides additional indexing
%commands, index entries can be created by issuing the \verb"\index"
%command. \index{dummy text|(}It is possible to create ranged index
%entries, which will encompass a span of text.\index{dummy text|)}
%To insert complex typographic material -- such as $\alpha$
%\index{alpha@$\alpha$} or \TeX{} \index{TeX@\TeX} --
%into the index, you need to specify a text string, which will
%determine how the entry will be sorted. It is also possible to
%create hierarchal entries. \index{vehicles!trucks}
%\index{vehicles!speed cars}

%index:
%  \makeatletter\thesis@blocks@clear\makeatother
%  \phantomsection %% Print the index and insert it into the
%  \addcontentsline{toc}{chapter}{\indexname} %% table of contents.
%  \printindex

\appendix %% Start the appendices.
\chapter{An appendix}
Here you can insert the appendices of your thesis.

\end{document}

%%%%%%%%%%%%%%%%%%%%%%%%%%%%%%%%%%%%%%%%%%%%%%%%%%%%%%%%%%%%%%%%%%%%
%% I, the copyright holder of this work, release this work into the
%% public domain. This applies worldwide. In some countries this may
%% not be legally possible; if so: I grant anyone the right to use
%% this work for any purpose, without any conditions, unless such
%% conditions are required by law.
%%%%%%%%%%%%%%%%%%%%%%%%%%%%%%%%%%%%%%%%%%%%%%%%%%%%%%%%%%%%%%%%%%%%

\documentclass[
  digital, %% This option enables the default options for the
           %% digital version of a document. Replace with `printed`
           %% to enable the default options for the printed version
           %% of a document.
  twoside, %% This option enables double-sided typesetting. Use at
           %% least 120 g/m² paper to prevent show-through. Replace
           %% with `oneside` to use one-sided typesetting; use only
           %% if you don’t have access to a double-sided printer,
           %% or if one-sided typesetting is a formal requirement
           %% at your faculty.
  table,   %% This option causes the coloring of tables. Replace
           %% with `notable` to restore plain LaTeX tables.
  lof,     %% This option prints the List of Figures. Replace with
           %% `nolof` to hide the List of Figures.
  lot,     %% This option prints the List of Tables. Replace with
           %% `nolot` to hide the List of Tables.
  %% More options are listed in the user guide at
  %% <http://mirrors.ctan.org/macros/latex/contrib/fithesis/guide/mu/fi.pdf>.
]{fithesis3}
%% The following section sets up the locales used in the thesis.
\usepackage[resetfonts]{cmap} %% We need to load the T2A font encoding
%\usepackage[T1,T2A]{fontenc}  %% to use the Cyrillic fonts with Russian texts.
\usepackage[
  main=english, %% By using `czech` or `slovak` as the main locale
                %% instead of `english`, you can typeset the thesis
                %% in either Czech or Slovak, respectively.
  english, german, russian, czech, slovak %% The additional keys allow
]{babel}        %% foreign texts to be typeset as follows:
%%
%%   \begin{otherlanguage}{german}  ... \end{otherlanguage}
%%   \begin{otherlanguage}{russian} ... \end{otherlanguage}
%%   \begin{otherlanguage}{czech}   ... \end{otherlanguage}
%%   \begin{otherlanguage}{slovak}  ... \end{otherlanguage}
%%
%% For non-Latin scripts, it may be necessary to load additional
%% fonts:
\usepackage{paratype}
\def\textrussian#1{{\usefont{T2A}{PTSerif-TLF}{m}{rm}#1}}
%%
%% The following section sets up the metadata of the thesis.
\thesissetup{
    date          = \the\year/\the\month/\the\day,
    university    = mu,
    faculty       = fi,
    type          = mgr,
    author        = Juraj S\'{i}\v{c},
    gender        = m,
    advisor       = Jan Strej\v{c}ek,
    title         = {Solving DQBF using BDDs},
    TeXtitle      = {Solving DQBF using BDDs},
    keywords      = {DQBF, BDD, solver},
    TeXkeywords   = {DQBF, BDD, solver},
    abstract      = {%
      This is the abstract of my thesis, which can

      span multiple paragraphs.
    },
    thanks        = {%
      These are the acknowledgements for my thesis, which can

      span multiple paragraphs.
    },
    bib           = bibliography.bib
    %% Uncomment the following line (by removing the % symbol at
    %% the beginning) and replace `assignment.pdf` with the
    %% filename of your scanned thesis assignment.
    % assignment    = assignment.pdf,
}
\usepackage{makeidx}      %% The `makeidx` package contains
\makeindex                %% helper commands for index typesetting.
%% These additional packages are used within the document:
\usepackage{paralist} %% Compact list environments
\usepackage{amsmath}  %% Mathematics
\usepackage{amsthm}
\usepackage{amsfonts}
\usepackage{url}      %% Hyperlinks
\usepackage{markdown} %% Lightweight markup
\usepackage{tabularx} %% Tables
\usepackage{tabu}
\usepackage{booktabs}
\usepackage{listings} %% Source code highlighting
\lstset{
  basicstyle      = \ttfamily,
  identifierstyle = \color{black},
  keywordstyle    = \color{blue},
  keywordstyle    = {[2]\color{cyan}},
  keywordstyle    = {[3]\color{olive}},
  stringstyle     = \color{teal},
  commentstyle    = \itshape\color{magenta},
  breaklines      = true,
}
\usepackage{floatrow} %% Putting captions above tables
\floatsetup[table]{capposition=top}
%% The following code fixes the rendering of BibLaTeX ISO 690
%% references in old TeX Live (such as the one at Overleaf).
\thesisload
\makeatletter
\def\thesis@biblatexiso@fix@package{iso-numeric.bbx}
\def\thesis@biblatexiso@fix@end{\relax}
\newif\ifthesis@biblatexiso@fix@
\thesis@biblatexiso@fix@false
\def\thesis@biblatexiso@fix@next#1,{%
  \def\thesis@biblatexiso@fix@current{#1}%
  \ifx\thesis@biblatexiso@fix@current\thesis@biblatexiso@fix@package
    \thesis@biblatexiso@fix@true
  \fi
  \ifx\thesis@biblatexiso@fix@current\thesis@biblatexiso@fix@end
    \expandafter
    \@gobble
  \fi
  \thesis@biblatexiso@fix@next
}
\expandafter\expandafter\expandafter\thesis@biblatexiso@fix@next\@filelist,\relax,
\ifthesis@biblatexiso@fix@
  \defbibenvironment{bibliography}
    {\list%
       {\MethodFormat}%
       {\setlength{\labelwidth}{\labelnumberwidth}%
        \setlength{\leftmargin}{\labelwidth}%
        \setlength{\labelsep}{\biblabelsep}%
        \addtolength{\leftmargin}{\labelsep}%
        \setlength{\itemsep}{\bibitemsep}%
        \setlength{\parsep}{\bibparsep}}%
        \renewcommand*{\makelabel}[1]{\hss##1}
        }%
    {\endlist}%
  {\item}%
\fi
\makeatother

%%%%%%%%%%%%%%%%%%%%%%%%%%%%%%%%%%%%%%%%%%%%%%%%%%%%%%%%%
%%%%%%%%%%%%%%%%%%%%%%%%%%%%%%%%%%%%%%%%%%%%%%%%%%%%%%%%%
%%%%%% The packages I use, my definitions, etc. %%%%%%%%%
%%%%%%%%%%%%%%%%%%%%%%%%%%%%%%%%%%%%%%%%%%%%%%%%%%%%%%%%%
%%%%%%%%%%%%%%%%%%%%%%%%%%%%%%%%%%%%%%%%%%%%%%%%%%%%%%%%%

\usepackage{mathtools}
\usepackage{braket}

%% We will define several mathematical sectioning commands.
\newtheorem{theorem}{Theorem}[section] %% The numbering of theorems
                               %% will be reset after each section.
\newtheorem{lemma}[theorem]{Lemma}         %% The numbering of lemmas
\newtheorem{corollary}[theorem]{Corollary} %% and corollaries will
                               %% share the counter with theorems.
\theoremstyle{definition}
\newtheorem{definition}{Definition}
\theoremstyle{remark}
\newtheorem*{remark}{Remark}

%%%%%%%%%%%%%%%%%%%%%%%%%%%%%%%%%%%%%%%%%%%%%%%%%%%%%%%%%
%%%%%%%%%%%%%%%%%%%%%%%%%%%%%%%%%%%%%%%%%%%%%%%%%%%%%%%%%

\begin{document}
\chapter*{Introduction}
\addcontentsline{toc}{chapter}{Introduction}
\begin{otherlanguage}{czech}
Říká se, že závěrečné práce jsou vyvrcholením studia a tak jsem se
rozhodl jednu také napsat. Pokud vše půjde podle plánu, odnesu si
na konci semestru diplom. Držte mi palce!
\end{otherlanguage}

\begin{otherlanguage}{slovak}
Hovorí sa, že záverečné práce sú vyvrcholením štúdia a tak som sa
rozhodol jednu tiež napísať. Ak všetko pôjde podľa plánu, odnesiem
si na konci semestra diplom. Držte mi palce!
\end{otherlanguage}

\begin{otherlanguage}{german}
Man munkelt, dass die Dissertation die Krönung der Ausbildung ist.
Deshalb habe ich mich beschlossen meine eigene zu schreiben. Wenn
alles gut geht, bekomme ich bald ein Diplom. Wünsch mir Glück!
\end{otherlanguage}

\begin{otherlanguage}{russian}\textrussian{%
Говорят, что тезис -- это кульминация обучения. Поэтому я и решил
написать собственный тезис. Если всё сработает по плану, я скоро
получу диплом. Желайте мне удачи!
}\end{otherlanguage}

\chapter{Theory}
popisat co je DQBF (ist od BF->QBF->DQBF?), neviem ci pridavat aj nonprenex, popisat BDDs, popisat ze co je nasim problemom

In this chapter we give needed ... theory for our problem... definition of dependency quantified Boolean formulas??? and BDDs?? citacie odkial vychadzam??

\section{Boolean Formulas}
In this section we will define dependency quantified Boolean formulas building up from Boolean formulas and adding quantifications followed by adding dependencies for these quantifications.

%definicia boolean formula (rekurzivna)

We first start with defining Boolean formulas. For these we need a set of variables $V = \set{x_1, \dots, x_n}$ for which we can define a valuation function $v\colon V \mapsto \{0,1\}$ that says whether each variable is true or false. Now we can define Boolean formulas as variables connected using logical operatives \emph{and} (${\land}$), \emph{or} (${\lor}$) and \emph{negation} (${\neg}$).

\begin{definition}
Let $V = \set{x_1, \dots, x_n}$ be a set of variables. \emph{Boolean formula} (BF) over $V$ is ...(a word??)... recursively defined as (has a form ?????)
\begin{itemize}
    \item $x_i$, where $x_i \in V$,
    \item $\phi_1 \land \phi_2$, where $\phi_1$ and $\phi_2$ are both Boolean formulas,
    \item $\phi_1 \lor \phi_2$, where $\phi_1$ and $\phi_2$ are both Boolean formulas,
    \item $\neg\phi$, where $\phi$ is a Boolean formula.
\end{itemize}
\end{definition}
%definicia semantiky BF (+splnitelnost)
Let $BF(V)$ be the set of all Boolean formulas over $V$ and $v\colon V \mapsto \set{0,1}$ a valuation over the set of variables $V$. We can extend this valuation to $BF(V)$, $v\colon BF(V) \mapsto \set{0,1}$ where
\begin{itemize}
    \item for $x_i \in V$ $v(x_i)$ stays the same
    \item $v(\phi_1 \land \phi_2) = 1$ if both $v(\phi_1) = 1$ and $v(\phi_2) = 1$, otherwise $v(\phi_1 \land \phi_2) = 0$,
    \item $v(\phi_1 \lor \phi_2) = 1$ if either $v(\phi_1) = 1$ or $v(\phi_2) = 1$, otherwise $v(\phi_1 \lor \phi_2) = 0$ and
    \item $v(\neg\phi) = 1$ if $v(\phi) = 0$ otherwise $v(\neg\phi) = 0$.
\end{itemize}
%We will abuse notation and use $v(\phi)$ for extended $v$????? 
%mozno napisat ako ze ten extend nejak oznacime (napr u_v) ale ak je to jasne tak abusneme notation a budeme pod $v(\phi)$ mysliet $u_v(\phi)$


An important thing for BFs is satisfiability. We say that BF $\phi$ over variables $V$ is satisfiable if there exists valuation $v$ over $V$ in which $\phi$ is true, that is $v(\phi) = 1$. Let $\phi_1, \phi_2 \in BF(V)$. We say that they are \emph{equivalent} ($\phi_1 = \phi_2$) if for all valuations $v$, $v(\phi_1) = v(\phi_2)$. If they are either both satisfiable or they are both unsatisfiable, we call them \emph{equisatisfiable}. vysetlit ze equisatisfiable znmmena ze tam mozu byt ine variables

priklad

%definovat negation normal form + cnf
\subsection{Conjuctive Normal Form}
To work with formulas algorithmically it is usually easier to have them in some special forms. The first one we define is called conjuctive normal form.
\begin{definition}
We say that boolean formula $\phi \in BF(V)$ is in conjuctive normal form (CNF) if there exist BFs $\psi_1,\dots,\psi_n \in BF(V)$ where
\[\phi = \psi_1 \land \dots \land \psi_n\]
and
\[\psi_i = l_{i1} \lor \dots \lor l_{i{m_i}}\]
where $i \in \set{1, \dots, n}$ and either $l_ij = x_ij$ or $l_ij = \neg x_ij$ where $x_ij \in V$.
\end{definition}
%it is a conjuction of disjuncts of literal where literal is either variable or negation of variable
For every BF $\phi$ there exists an equivalent formula in CNF which can be exponentially larger than $\phi$. However, there exists a construction that by adding some new variables results in an equisatisfiable BF $\phi_{CNF}$ which is only polynomially larger than the original BF $\phi$ (citedacotu).
%TODO explain clause, literal...

\subsection{Negative Normal Form????}
toto asi zadefinujem pre QBF/DQBF ak vobec

\section{Quantified Boolean Formulas}

Having defined BFs, we can move to the next step on our way to defining DQBF by adding quantifiers. We add existential (${\exists}$) and universal (${\forall}$) quantifiers which will be bounded to variables. %Existential quantifier $\exists x_i$ will then tell us that at least for one of $x_i = 0$, $x_i = 1$ formula holds, while for universal quantifier $\forall x_i$ for both $x_i = 0$ and $x_i = 1$ formula holds.

%definicia QBF (ak budem robit aj non-prenex DQBF tak rekurzivna + nahradzovanie, inak prenex)
\begin{definition}
Let $V$ be a set of variables and $\phi \in BF(V)$ a BF over $V$. A \emph{quantified Boolean formula} (QBF) $\psi$ has the form
\[\psi \coloneqq Q_1 x_1 Q_2 x_2 \dots Q_n x_n \phi\]
where $Q_i \in \set{{\exists}, {\forall}}$ for all $i \in \set{1,2,\dots,n}$.
\end{definition}

%definicia semantiky (+splnitelnost)
Let $QBF(V)$ be the set of all quantified Boolean formulas over $V$ and $v\colon V \mapsto \set{0,1}$ a valuation over the set of variables $V$. Again, we can extend $v$ to $QBF(V)$ where 
\begin{itemize}
    \item for $\phi \in BF(V)$, $v(\phi)$ behaves like $v$ extended to $BF(V)$ as explained in section TODO
    \item $v(\forall x Q_2 x_2 \dots Q_n x_n \phi) = 1$ if both $v(Q_2 x_2 \dots Q_n x_n \phi) = 1$ and ... TODO definovat nahrazdovanie + doplnit do definicie BF 1 a 0 konstanty
\end{itemize}
%mozno pisat skor ze pre \psi z QBF, v(psi) je definovana pre tvar \psi = itemize ...
%definovat bounded
Notice that in this definition of semantics the variables that are bounded with some quantifier do not care what they are evaluated to. That is priklad It can be seen that if addded $\exists x$ to unbounded variables $x$ to the beginning of any QBF, we would not need any valuation etcetc

We can also define satisfiability on QBF, similarly to BF. We say that QBF $\psi$ is satisfiable if there exists valuation $v$ such that $v(\psi) = 1$.

priklad

\section{Dependency Quantified Boolean Formulas}
Quantifiers have expanded BFs quite considerably (mozno nieco ako expressibality???). However, there is still one drawback of QBFs - quantified variable $x$ depends on all variables that are quantified before $x$. The problem with this dependency relation is that it is given by how quantifiers follow each other. vysvetlit na priklade z predchadzajucej sekcie But what if we want nonlinear dependencies? And what would they even mean? This is where dependency quantified Boolean formulas will shine!

%definicia DQBF -ak robim nonprenex tak zacat s prenex ale vysvetlit ze pre jednoduchost zacneme s tym + semantika prenex + priklady
\begin{definition}
Let $V = \set{x_1, \dots, x_n, y_1, \dots, y_m}$ be a set of variables and $\phi \in BF(V)$ a BF over $V$. A \emph{dependency quantified Boolean formula} (DQBF) $\psi$ has the form
\[\psi \coloneqq \forall x_1 \dots \forall x_n \exists (D_1) y_1 \dots \exists (D_m) y_m \phi\]
where $D_i \subseteq \set{x_1, \dots, x_n}$, $i \in \set{1,\dots,m}$, is a \emph{dependency sets} of variable $y_i$.
\end{definition}
Let use an example to explain the meaning behind dependency sets. Let 
\[\psi = \forall x_1 \forall x_2 \exists (x1) y_1 \exists (x_2) y_2 : (x_1 \land x_2) \iff (y1 \iff y2)\]
be a DQBF. Both $y_1$ and $y_2$ depend only on one variable, $x_1$ and $x_2$ respectively. TODO + citovat odkial som vzal priklad

tu dat nonprenex ak robim, semantika + priklady + dokazy o nahradzovani a eliminovani kvantifikatorov



\section{Binary Decision Diagrams}
Having defined DQBFs we can now turn our attention to their representation. We have already defined some normal forms in section blabla, but algorithmically, working with them in text form would not be very effective (why?). For this reason binary decision diagrams were introduced. These structures allow for ??effective?? represantion of BFs and operations on them???

\begin{definition}
???
\end{definition}

nieco o ROBDD, ze pre kazde BF existuje prave jedno unikatne pre kazde zoradenie + ze vyber zoradenia je dolezity + ze da sa dynamicky preukladat???

\chapter{The Problem and Current Solutions???? state of the art}
%problem - vytvorit algoritmus ktory zisti ze ci dana DQBF formula je splnitelna alebo nie pomocou BDDcok alebo jej podobnym strukturam (mozno dat do teorie)

Having laid out all the necessary theory we can now define the problem we are trying to solve and current implementations solving it.

\section{The Problem of Satisfiability}
Let $\psi$ be DQBF. The problem we are trying to solve is to tell whether this formula is satisfiable, that is if there exists model in which it is true. For BF, this problem belongs to NP-complete class of problems cite. If we move to QBF, the problem gets harder and belongs to PSPACE-complete class. As was shown in CITE, this problem for DQBF belongs in NEXPTIME-complete class of problems, that is it is even harder.

\section{State of the Art}
%ake solvre existuju (vybrat tie s ktorymi budem porovnavat?), trocha viac popisat HQS a ze ako funguje aj HQSpre (lebo ho pouzivam) - ale fakt len zopar veci spomenut, mozno aj to ze mozme pouzit vytvaranie log. gates pretoze to menim do BDD ako to robia v HQS
In this section we give an overview of existing solvers which tackle the problem. An overview of existing solvers can be found in~\cite{DQBFStateOfArtTalk,DQBFStateOfArt} (mozno tu dat aj ten talk z 2015) that is why we just give a short updated overview here (short for DQDPLL, iDQ, iProver and dCAQE). However we go into more details for solver HQS, because the methods we used for developing our solver are based on the workings of this solver.
%current solutions, that already exist

%nieco o tom ze vacsina funguje na CNF, ale HQS a moje bude na NNF

%fast DQBF refutation - ze premienaju na QBF abo daco (bunsat)

\subsection{First Solution - DQDPLL}
The first solver that tackled the satisfiability problem for DQBF was based on DPLL(TODO cite) algorithm which is succesfully used for BF and QBF solvers(TODO cite??). This algorithm works on formulas in CNF by  recursively trying to set variables to either true or false while checking if the disjuncts in CNF stay consistent. ??popisat nejak lepsie??? An adaptation called DQDPLL~\cite{DPLLalgorithm} was introduced for DQBF, however it did not result in competitive solver.

\subsection{Instatiation - iDQ??(+iProver??)}
The first ??efficient?? DQBF solver (called iDQ~\cite{iDQandDQDIMACS}) is based on instantiation technique used for solving Effectively Propositional Logic (EPR)(cite). This solver also works on formulas in CNF where in each step of the algorithm it tries to create a BF that is an overapproximation of input DQBF by instanciating each clause .... This BF is then checked if it is satisfiable, if it is not, then the input DQBF is not satisfiable. If it is, they check whether resulting valuation is valid for the input DQBF. If it is not, then it is used to create more clause instances which are then used to refine this overapproximation.

However, EPR can be also used direclty. Because same as DQBF it belongs to NEXPTIME class, there exists a polynomial ... from DQBF to EPR (cite). This can be used by some EPR solver to transform DQBF to EPR and then solve resulting EPR. The EPR solver iProver uses this technique to solve DQBF.

mozno povedat ze iProver sa ukazal lepsi

%iProver - just turn DQBF to EPR instance and run EPR solver

%is also NEXPTIME-complete, which means there exists a polynomial reduction from DQBF to EPR. The iDQ solver can then use techniques used for solving EPR logic but reduced to the more specific case of DQBF.

\subsection{Abstraction Refinement - dCAQE}
\cite{dCAQE} zalozene na abstraction refinement (viacmenej iDQ je tiez abstraction refinement). rozdelia si kvantifikatory na nejake urovne podla zavislosti a pouzitim 

\subsection{Quanfier Elimination - HQS}
The next solver HQS is based on quantifier elimination. This solver's basic premise is easy - it iteratively chooses some universal quantifier that is eliminated using theorem ???? and then remove all existential quantifiers that are dependent on all leftover universal quantifiers (theroem???).
%nejak vysvetlit ze pouzivaju AIG

In (citeprvynaivnyalgclanok) the authors introduced the basic algorithm for solving DQBF in this way. ...vlozit algoritmus???... They first choose the blabla. The universal variable is chosen based on the number of existential ones that depend on it - they choose the maximal.

Following this technique, in (citeqe) they introduce solver HQS which was enhanced by using QBF solver as subprocedure. They still eliminate quantifiers in similar fashion, but now they do it until the formula results in QBF form. On this QBF they then run already existing solver for QBF called AIG-solver(cite) which is much more optimized and effective. Universal quantifiers to eliminate are chosen in the beginning in such a way that the number of universal eliminations is small as possible while still the resulting formula is QBF. For this they build a dependency graph where nodes are existential variables and they are connected only if the dependency sets are not subsets. They noticed that if this graph is acyclic, the formula can be seen as QBF and so they solve the problem of finding minimum set of universal variables that when they are removed the graph is acyclic. For this they use MaxSAT????

This was good and all, but in (depelim) they zaoberat sa whether it is possible to remove only some universal variables from dependency sets. They showed it is possible ....tu asi dat cely ten teorem ak som ho nedal do teorie...

Last but not least in (quantlocalization) the authors have given  a theorethical foundations for non-prenex DQBFs and a possibility to push qunatifiers inside formulas. They also showed how to remove existential quantifiers inside formulas.????????????

\subsection{HQSpre}
HQS solver has a succesfull run soadfnosue??? winning tu sutaz dvakrat. However, the solver partially owes its success to the effeciency of the prepropcessor. The preprocessor HQSpre is able to solve a lot of DQBFs from the go, without need for using more complex techniques. First introduced in ... it took many techniques used for QBF preprocessing and ...it to the DQBF case. asipopisat ake?? + mozno spomenut ten approximations??

% vysvetlit prve dve clanky, ze ide o jednoducho univ. expansion -> exist. elimination pouzitim AIG a pridanim nejakych zaujimavych drobnosti sa vytvori QBF ktory sa hodi do nejakeho existujuceho solvru (pridat obrazok toho cyklu + naivny alg.), tu spomenut aj preprocessor ze je vysvetleny niekde
% potom vysvetlit ze je dolezity spravny vyber univ. kv. na expansion a vysvetlit ze v naivnom alg. to je jednoduche proste tie na ktore zavisia najviac, v tych dalsich clankoch vytvoreny dependency graf ktory sa nejak riesi, neskor sa odstranuje priamo jedna zavislost pre exist. kvantifikatory
% potom zacat hovorit o tom ze ako pchat kvantifikatory do vnutra, kde to nejak zrobili
% vysvetlit preprocessor (hlavne gate reduction alebo jak to nazvat)
% spomenut asi aj overapproximations

\chapter{Implementation/my solution??}

\chapter{Experimental results}
grafygrafygrafy

\chapter{These are}
\section{the available}
\subsection{sectioning}
\subsubsection{commands.}
\paragraph{Paragraphs and}
\subparagraph{subparagraphs are available as well.}
Inside the text, you can also use unnumbered lists,
\begin{itemize}
  \item such as
  \item this one
  \begin{itemize}
    \item     and they can be nested as well.
    \item[>>] You can even turn the bullets into something fancier,
    \item[\S] if you so desire.
  \end{itemize}
\end{itemize}
Numbered lists are
\begin{enumerate}
  \item very
  \begin{enumerate}
    \item similar
  \end{enumerate}
\end{enumerate}
and so are description lists:
\begin{description}
  \item[Description list]
    A list of terms with a description of each term
\end{description}
The spacing of these lists is geared towards paragraphs of text.
For lists of words and phrases, the \textsf{paralist} package
offers commands
\begin{compactitem}
  \item that
  \begin{compactitem}
    \item are
    \begin{compactitem}
      \item better
      \begin{compactitem}
        \item suited
      \end{compactitem}
    \end{compactitem}
  \end{compactitem}
\end{compactitem}
\begin{compactenum}
  \item to
  \begin{compactenum}
    \item this
    \begin{compactenum}
      \item kind of
      \begin{compactenum}
        \item content.
      \end{compactenum}
    \end{compactenum}
  \end{compactenum}
\end{compactenum}
The \textsf{amsthm} package provides the commands necessary for the
typesetting of mathematical definitions, theorems, lemmas and
proofs.

\begin{theorem}
  This is a theorem that offers a profound insight into the
  mathematical sectioning commands.
\end{theorem}
\begin{theorem}[Another theorem]
  This is another theorem. Unlike the first one, this theorem has
  been endowed with a name.
\end{theorem}
\begin{lemma}
  Let us suppose that $x^2+y^2=z^2$. Then
  \begin{equation}
    \biggl\langle u\biggm|\sum_{i=1}^nF(e_i,v)e_i\biggr\rangle
    =F\biggl(\sum_{i=1}^n\langle e_i|u\rangle e_i,v\biggr).
  \end{equation}
\end{lemma}
\begin{proof}
  $\nabla^2 f(x,y)=\frac{\partial^2f}{\partial x^2}+
   \frac{\partial^2f}{\partial y^2}$.
\end{proof}
\begin{corollary}
  This is a corollary.
\end{corollary}
\begin{remark}
  This is a remark.
\end{remark}

\chapter{Floats and references}
\begin{figure}
  \begin{center}
    %% PNG and JPG images can be inserted into the document as well,
    %% but their resolution needs to be adequate. The minimum is
    %% about 100 pixels per 1 centimeter or 300 pixels per 1 inch.
    %% That means that a JPG or PNG image typeset at 4 × 4 cm should
    %% be 400 × 400 px large at the bare minimum.
    %%
    %% The optimum is about 250 pixels per 1 centimeter or 600
    %% pixels per 1 inch. That means that a JPG or PNG image typeset
    %% at 4 × 4 cm should be 1000 × 1000 px large or larger.
    \includegraphics[width=4cm]{fithesis/logo/mu/fithesis-base.pdf}
  \end{center}
  \caption{The logo of the Masaryk University at 40\,mm}
  \label{fig:mulogo1}
\end{figure}

\begin{figure}
  \begin{center}
    \begin{minipage}{.66\textwidth}
      \includegraphics[width=\textwidth]{fithesis/logo/mu/fithesis-base.pdf}
    \end{minipage}
    \begin{minipage}{.33\textwidth}
      \includegraphics[width=\textwidth]{fithesis/logo/mu/fithesis-base.pdf} \\
      \includegraphics[width=\textwidth]{fithesis/logo/mu/fithesis-base.pdf}
    \end{minipage}
  \end{center}
  \caption{The logo of the Masaryk University at $\frac23$ and
    $\frac13$ of text width}
  \label{fig:mulogo2}
\end{figure}

\begin{table}
  \begin{tabularx}{\textwidth}{lllX}
    \toprule
    Day & Min Temp & Max Temp & Summary \\
    \midrule
    Monday & $13^{\circ}\mathrm{C}$ & $21^\circ\mathrm{C}$ & A
    clear day with low wind and no adverse current advisories. \\
    Tuesday & $11^{\circ}\mathrm{C}$ & $17^\circ\mathrm{C}$ & A
    trough of low pressure will come from the northwest. \\
    Wednesday & $10^{\circ}\mathrm{C}$ &
    $21^\circ\mathrm{C}$ & Rain will spread to all parts during the
    morning. \\
    \bottomrule
  \end{tabularx}
  \caption{A weather forecast}
  \label{tab:weather}
\end{table}

The logo of the Masaryk University is shown in Figure
\ref{fig:mulogo1} and Figure \ref{fig:mulogo2} at pages
\pageref{fig:mulogo1} and \pageref{fig:mulogo2}. The weather
forecast is shown in Table \ref{tab:weather} at page
\pageref{tab:weather}. The following chapter is Chapter
\ref{chap:matheq} and starts at page \pageref{chap:matheq}.
Items \ref{item:star1}, \ref{item:star2}, and
\ref{item:star3} are starred in the following list:
\begin{compactenum}
  \item some text
  \item some other text
  \item $\star$ \label{item:star1}
  \begin{compactenum}
    \item some text
    \item $\star$ \label{item:star2}
    \item some other text
    \begin{compactenum}
      \item some text
      \item some other text
      \item yet another piece of text
      \item $\star$ \label{item:star3}
    \end{compactenum}
    \item yet another piece of text
  \end{compactenum}
  \item yet another piece of text
\end{compactenum}
If your reference points to a place that has not yet been typeset,
the \verb"\ref" command will expand to \textbf{??} during the first
run of
\texttt{pdflatex \jobname.tex}
and a second run is going to be needed for the references to
resolve. With online services -- such as Overleaf -- this is
performed automatically.

\chapter{Mathematical equations}
\label{chap:matheq}
\TeX{} comes pre-packed with the ability to typeset inline
equations, such as $\mathrm{e}^{ix}=\cos x+i\sin x$, and display
equations, such as \[
  \mathbf{A}^{-1} = \begin{bmatrix}
  a & b \\ c & d \\
  \end{bmatrix}^{-1} =
  \frac{1}{\det(\mathbf{A})} \begin{bmatrix}
  \,\,\,d & \!\!-b \\ -c & \,a \\
  \end{bmatrix} =
  \frac{1}{ad - bc} \begin{bmatrix}
  \,\,\,d & \!\!-b \\ -c & \,a \\
  \end{bmatrix}.
\] \LaTeX{} defines the automatically numbered \texttt{equation}
environment:
\begin{equation}
  \gamma Px = PAx = PAP^{-1}Px.
\end{equation}
The package \textsf{amsmath} provides several additional
environments that can be used to typeset complex equations:
\begin{enumerate}
  \item An equation can be spread over multiple lines using the
    \texttt{multline} environment:
    \begin{multline}
      a + b + c + d + e + f + b + c + d + e + f + b + c + d + e +
f \\
      + f + g + h + i + j + k + l + m + n + o + p + q
    \end{multline}

  \item Several aligned equations can be typeset using the
    \texttt{align} environment:
    \begin{align}
              a + b &= c + d     \\
                  u &= v + w + x \\[1ex]
      i + j + k + l &= m
    \end{align}

  \item The \texttt{alignat} environment is similar to
    \texttt{align}, but it doesn't insert horizontal spaces between
    the individual columns:
    \begin{alignat}{2}
      a + b + c &+ d       &   &= 0 \\
              e &+ f + g   &   &= 5
    \end{alignat}

  \item Much like chapter, sections, tables, figures, or list
    items, equations -- such as \eqref{eq:first} and
    \eqref{eq:mine} -- can also be labeled and referenced:
    \begin{alignat}{4}
      b_{11}x_1 &+ b_{12}x_2  &  &+ b_{13}x_3  &  &             &
        &= y_1,                   \label{eq:first} \\
      b_{21}x_1 &+ b_{22}x_2  &  &             &  &+ b_{24}x_4  &
        &= y_2. \tag{My equation} \label{eq:mine}
    \end{alignat}

  \item The \texttt{gather} environment makes it possible to
    typeset several equations without any alignment:
    \begin{gather}
      \psi = \psi\psi, \\
      \eta = \eta\eta\eta\eta\eta\eta, \\
      \theta = \theta.
    \end{gather}

  \item Several cases can be typeset using the \texttt{cases}
    environment:
    \begin{equation}
      |y| = \begin{cases}
        \phantom-y & \text{if }z\geq0, \\
                -y & \text{otherwise}.
      \end{cases}
    \end{equation}
\end{enumerate}
For the complete list of environments and commands, consult the
\textsf{amsmath} package manual\footnote{
  See \url{http://mirrors.ctan.org/macros/latex/required/amslatex/math/amsldoc.pdf}.
  The \texttt{\textbackslash url} command is provided by the
  package \textsf{url}.
}.

\chapter{\textnormal{We \textsf{have} \texttt{several} \textsc{fonts}
  \textit{at} \textbf{disposal}}}
The serified roman font is used for the main body of the text.
\textit{Italics are typically used to denote emphasis or
quotations.} \texttt{The teletype font is typically used for source
code listings.} The \textbf{bold}, \textsc{small-caps} and
\textsf{sans-serif} variants of the base roman font can be used to
denote specific types of information.

\tiny We \scriptsize can \footnotesize also \small change \normalsize
the \large font \Large size, \LARGE although \huge it \Huge
is \huge usually \LARGE not \Large necessary.\normalsize

A wide variety of mathematical fonts is also available, such as: \[
  \mathrm{ABC}, \mathcal{ABC}, \mathbf{ABC}, \mathsf{ABC},
  \mathit{ABC}, \mathtt{ABC}
\] By loading the \textsf{amsfonts} packages, several additional
fonts will become available: \[
  \mathfrak{ABC}, \mathbb{ABC}
\] Many other mathematical fonts are available\footnote{
  See \url{http://tex.stackexchange.com/a/58124/70941}.
}.

\chapter{Using lightweight markup}
\shorthandoff{-}
\begin{markdown*}{%
  hybrid,
  definitionLists,
  footnotes,
  inlineFootnotes,
  hashEnumerators,
  fencedCode,
  citations,
  citationNbsps,
}

If you decide that \LaTeX{} is too wordy for some parts of your
document, there are [packages](https://www.ctan.org/pkg/markdown
"Markdown") that allow you to use more lightweight markup next
to it.

 ![logo](fithesis/logo/mu/fithesis-base.pdf "The logo of the
  Masaryk University")

This is a bullet list. Unlike numbered lists, bulleted lists
contain an **unordered** set of bullet points. When a bullet point
contains multiple paragraphs, the list is typeset as follows:

  * The first item of a bullet list

    that spans several paragraphs,
  * the second item of a bullet list,
  * the third item of a bullet list.

When none of the bullet points contains multiple paragraphs, the
list has a more compact form:

  * The first item of a bullet list,
  * the second item of a bullet list,
  * the third item of a bullet list.

Unlike a bulleted list, a numbered list implies chronology or
ordering of the bullet points. When a bullet point
contains multiple paragraphs, the list is typeset as follows:

  1. The first item of an ordered list

     that spans several paragraphs,
  2. the second item of an ordered list,
  3. the third item of an ordered list.
  #. If you are feeling lazy,
  #. you can use hash enumerators as well.

When none of the bullet points contains multiple paragraphs, the
list has a more compact form:

  6. The first item of an ordered list,
  7. the second item of an ordered list,
  8. the third item of an ordered list.

Definition lists are used to provide definitions of terms. When
a definition contains multiple paragraphs, the list is typeset
as follows:

Term 1

:   Definition 1

*Term 2*

:   Definition 2

        Some code, part of Definition 2

    Third paragraph of Definition 2.

When none of the bullet points contains multiple paragraphs, the
list has a more compact form:

Term 1
:   Definition 1
*Term 2*
:   Definition 2

Block quotations are used to include an excerpt from an external
document in way that visually clearly separates the excerpt from
the rest of the work:

> This is the first level of quoting.
>
> > This is nested blockquote.
>
> Back to the first level.

Footnotes are used to include additional information to the
document that are not necessary for the understanding of the main
text. Here is a footnote reference^[Here is the footnote.] and
another.[^longnote]

[^longnote]: Here's one with multiple blocks.

    Subsequent paragraphs are indented to show that they
belong to the previous footnote.

        Some code

    The whole paragraph can be indented, or just the first
    line.  In this way, multi-paragraph footnotes work like
    multi-paragraph list items.

Citations are used to provide bibliographical references to other
documents. This is a regular citation~[@borgman03, p. 123]. This is
an in-text citation: @borgman03\. You can also cite several authors
at once using both regular~[see @borgman03, p. 123; @greenberg98,
sec.  3.2; and @thanh01] and in-text citations: @borgman03 [p.123;
@greenberg98, sec. 3.2; @thanh01].

Code blocks are used to include source code listings into the
document:

    #include <stdio.h>
    #include <unistd.h>
    #include <sys/types.h>
    #include <sys/wait.h>
    // This is a comment
    int main(int argc, char **argv)
    {
        while (--c > 1 && !fork());
        sleep(c = atoi(v[c]));
        printf("%d\n", c);
        wait(0);
        return 0;
    }

There is an alternative syntax for code blocks that allows you to
specify additional information, such as the language of the source
code. This information can be used for syntax highlighting:

``` sh
#!/bin/sh
fac() {
  if [ "$1" -leq 1 ]; then
    echo 1
  else
    echo $(("$1" * fac $(("$1" - 1))))
  fi
}
``````````````

~~~~~~ Ruby
# Here's a way to empty an array.
joe = [ 'eggs.', 'some', 'break', 'to', 'Have' ]
print(joe.pop, " ") while joe.size > 0
print "\n"
~~~~~~

\end{markdown*}
\shorthandon{-}

\chapter{Inserting the bibliography}
After linking a bibliography data\-base files to the document using
the \verb"\"\texttt{thesis\discretionary{-}{}{}setup\{bib\discretionary{=}{=}{=}%
\{\textit{file1},\textit{file2},\,\ldots\,\}\}} command, you can
start citing the entries. This is just dummy text
\parencite{borgman03} lightly sprinkled with citations
\parencite[p.~123]{greenberg98}. Several sources can be cited at
once: \cite{borgman03,greenberg98,thanh01}.
\citetitle{greenberg98} was written by \citeauthor{greenberg98} in
\citeyear{greenberg98}. We can also produce \textcite{greenberg98}%
\ or %% Let us define a compound command:
\def\citeauthoryear#1{(\textcite{#1},~\citeyear{#1})}%
\citeauthoryear{greenberg98}%
. The full bibliographic citation is:
\emph{\fullcite{greenberg98}}. We can easily insert a bibliographic
citation into the footnote\footfullcite{greenberg98}.

The \verb"\nocite" command will not generate any
output\nocite{muni}, but it will insert its arguments into
the bibliography. The \verb"\nocite{*}" command will insert all the
records in the bibliography database file into the bibliography.
Try uncommenting the command
%% \nocite{*}
and watch the bibliography section come apart at the seams.

When typesetting the document for the first time, citing a
\texttt{work} will expand to [\textbf{work}] and the
\verb"\printbibliography" command will produce no output. It is now
necessary to generate the bibliography by running \texttt{biber
\jobname.bcf} from the command line and then by typesetting the
document again twice. During the first run, the bibliography
section and the citations will be typeset, and in the second run,
the bibliography section will appear in the table of contents.

The \texttt{biber} command needs to be executed from within the
directory, where the \LaTeX\ source file is located. In Windows,
the command line can be opened in a directory by holding down the
\textsf{Shift} key and by clicking the right mouse button while
hovering the cursor over a directory.  Select the \textsf{Open
Command Window Here} option in the context menu that opens shortly
afterwards.

With online services -- such as Overleaf -- or when using an
automatic tool -- such as \LaTeX MK -- all commands are executed
automatically. When you omit the \verb"\printbibliography" command,
its location will be decided by the template.

  \printbibliography[heading=bibintoc] %% Print the bibliography.

\chapter{Inserting the index}
After using the \verb"\makeindex" macro and loading the
\texttt{makeidx} package that provides additional indexing
commands, index entries can be created by issuing the \verb"\index"
command. \index{dummy text|(}It is possible to create ranged index
entries, which will encompass a span of text.\index{dummy text|)}
To insert complex typographic material -- such as $\alpha$
\index{alpha@$\alpha$} or \TeX{} \index{TeX@\TeX} --
into the index, you need to specify a text string, which will
determine how the entry will be sorted. It is also possible to
create hierarchal entries. \index{vehicles!trucks}
\index{vehicles!speed cars}

After typesetting the document, it is necessary to generate the
index by running
\begin{center}%
  \texttt{texindy -I latex -C utf8 -L }$\langle$\textit{locale}%
  $\rangle$\texttt{ \jobname.idx}
\end{center}
from the command line, where $\langle$\textit{locale}$\rangle$
corresponds to the main locale of your thesis -- such as
\texttt{english}, and then typesetting the document again.

The \texttt{texindy} command needs to be executed from within the
directory, where the \LaTeX\ source file is located. In Windows,
the command line can be opened in a directory by holding down the
\textsf{Shift} key and by clicking the right mouse button while
hovering the cursor over a directory. Select the \textsf{Open Command
Window Here} option in the context menu that opens shortly
afterwards.

With online services -- such as Overleaf -- the commands are
executed automatically, although the locale may be erroneously
detected, or the \texttt{makeindex} tool (which is only able to
sort entries that contain digits and letters of the English
alphabet) may be used instead of \texttt{texindy}. In either case,
the index will be ill-sorted.

  \makeatletter\thesis@blocks@clear\makeatother
  \phantomsection %% Print the index and insert it into the
  \addcontentsline{toc}{chapter}{\indexname} %% table of contents.
  \printindex

\appendix %% Start the appendices.
\chapter{An appendix}
Here you can insert the appendices of your thesis.

\end{document}
